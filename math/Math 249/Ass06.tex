\documentclass[11pt]{article}
\usepackage{geometry}                % See geometry.pdf to learn the layout options. There are lots.
\geometry{letterpaper}                   % ... or a4paper or a5paper or ... 
%\geometry{landscape}                % Activate for for rotated page geometry
\usepackage[parfill]{parskip}    % Activate to begin paragraphs with an empty line rather than an indent
\usepackage{graphicx}
\usepackage{amssymb,amsmath}
\usepackage{fullpage}
\usepackage{epstopdf}
\DeclareGraphicsRule{.tif}{png}{.png}{`convert #1 `dirname #1`/`basename #1 .tif`.png}

\title{MATH 249 - Assignment 6}
\author{Adrian Petrescu (\#20240298)}
%\date{}                                           % Activate to display a given date or no date

\begin{document}
\maketitle
%\section{}
%\subsection{}

\textbf{1. Let $G$ be a simple planar graph that does not contain a triangle. Show that the vertices of $G$ can be properly coloured with at most 4 colours.}

We recall the result for planar graphs that $q\leq3p-6$, from Euler's formula. We begin by proving a stronger result for planar graphs that do not contain triangles, namely that $q\leq2p-4$. Substituting, we get $3p-3q+6q\leq6\implies$

Well, by Euler's formula, we know $p-q+r=2$ so $2p-2q+2r=4$; by the handshake lemma for faces we usually get that $3r\leq2q$; however, this time we know that there are no triangles, so each face is bound by at least 4 edges, not 3; so $2r\leq q$ gives us $2p-2q+q\geq4\implies q\leq2p-4$, which is what we were trying to prove.

Now we use this fact to show that all planar graphs with no triangles have a vertex of degree less than or equal to 3. We proceed similarly to the technique for showing that general planar graphs have a vertex of degree less than or equal to 5.

Let $n_i$ be the number of vertices of degree $i$ in $G$. We know from the handshake lemma that $2q=n_1+2n_2+3n_3+\cdots$. Also, $p=n_1+n_2+n_3+\cdots$. Then from the result above, we also have $2q\leq4p-8$. We combine these two facts to get
\begin{align*}
	n_1+2n_2+3n_3+\cdots\leq&4(n_1+n_2+n_3+\cdots)-8\\
	8+n_1+2n_2+3n_3+4n_4+5n_5+\cdots\leq&4n_1+4n_2+4n_3+4n_4+4n_5+\cdots\\
	8+n_5+2n_6+3n_7+\cdots\leq&3n_1+2n_2+n_3
	\end{align*}  
Since the quantity on the left is clearly positive, and the quantity on the right is bigger, it must also be positive; so there are always some vertices of degree less than or equal to 3 in $G$.

But now we are essentially done. Consider one of those vertices, call it $\vec v_0$. By induction, we know $G\backslash v_0$ is 4-colourable, so it only remains to insert $v_0$ back in and make sure it is coloured properly. But it is of degree at most 3, and we have 4 colours to choose from, so we will surely have a colour at our disposal to paint $v_0$ with; thus $G$ is 4-colourable, and we are done.

\textbf{2. Let $G$ be a bipartite graph, and let $\Delta$ by the maximum degree of any vertex in $G$. Show that $G$ has a matching with at least $q/\Delta$ edges. Also, give an example of a non-bipartite graph for which this lower bound on the size of a matching is not true.}

Since $G$ is a bipartite graph, Konig's theorem holds which tells us that $\max{|M|}=\min{|C|}$. Assume by contradiction that $G$ does not have a matching with at least $q/\Delta$ edges; that means $\max{|M|}<q/\Delta$, which therefore means that $\min|C|<q/\Delta$.

We draw our bipartite partition of the vertices into groups $A$ and $B$. Take some edge $v_0$ with degree $\Delta$. (One must exist). We know it must be part of the minimal cover since it exhausts more of the edges than any other vertex (except perhaps other vertices with degree $\Delta$). Thus we add it to our cover $C$ and count $\Delta$ of the $q$ edges as "exhausted". We continue adding vertices in decreasing order thus ensuring minimality. At the end, we know that each vertex occupied at most $\Delta/q$ edges. Thus in total we have
\[
\sum_{i=1}^{|C|}\frac{\Delta}{q}\geq\frac{q}{q}\implies|C|\frac{\Delta}{q}\geq1\implies|C|\geq\frac{q}{\Delta}
\]

But this is a contradiction, since we always chose the vertices of highest degree first. Thus a minimal cover less than $q/\Delta$ must have been impossible, thus leading to a contradiction. So it also follows that $G$ has a matching with at least $q/\Delta$ edges.

As a counterexample for when $G$ is not bipartite, consider the simple triangle. It has 3 edges and the maximum degree is $4$, so $q/\Delta>1$. But you can only match one edge on the triangle before you run out, so it does not satisfy the requirement.

\textbf{3. Let $G$ be a bipartite graph with bipartition $(A,B)$ such that $|A|=|B|$. Assume that the Hall condition holds with strict inequality for every subset of $A$; that is, assume that for all $S\subseteq A,|N(S)|>|S|$. Show that for every edge $e\in E$ there is a perfect matching of $G$ that contains $e$.}

As per the usual proof of Hall's Theorem, split $A$ and $B$ into equally-sized partitions. Consider the arbitrary edge $e\in E$ that we wish to include in a perfect matching of $G$. Say that $e=\{a_0,b_0\}$. Then the graph $G\backslash a_0\backslash b_0$ is another bipartite graph with $|A|=|B|$. Now, we know that $|S|<|N(S)|$ for any subset, including those which did not include $a_0$. Thus all the subsets $N(S)$, which may have included $b_0$, may lose up to one vertex but because the inequality was strict, this loss is not enough to make the resulting $G\backslash a_0\backslash b_0$ lose the regular (soft) Hall condition. Thus Hall's theorem applies to $G\backslash a_0\backslash b_0$, and so it has a perfect matching. But now, if we bring back $a_0$ and $b_0$, they are the only unmatched pair in the graph; thus we add the edge $e=\{a_0,b_0\}$ into the matching, and we have a perfect matching. Since our choice of $e$ was arbitrary, we could apply this to any edge in $G$ to ensure it is included in the perfect matching, which is what we were trying to prove; thus we are done.

\textbf{4. Draw an example of a $3$-regular graph that does not have a perfect matching.}

The graph is drawn by hand and attached to a piece of paper that was handed in with this assignment.

\textbf{5. Use the Matrix-Tree Theorem to show that for all $n\geq1$, the complete graph $K_n$ has $n^{n-2}$ spanning trees.}

It is clear that $K_n$ is $(n-1)$-regular, so the main diagonal will have entirely $n-1$ across it. Moreover, every vertex is connected to every other vertex by precisely on edge in $K_n$, so all other non-diagonal entries will be $-1$ in the Laplacian matrix. So we have so far
\[
Q=\begin{bmatrix}
n-1 & -1 & -1 & -1 & \ldots & -1 & -1 \\
-1 & n-1 & -1 & -1 & \ldots & -1 & -1 \\
-1 & -1 & n-1 & -1 & \ldots & -1 & -1 \\
-1 & -1 & -1 & n-1 & \ldots & -1 & -1 \\
\vdots & \vdots & \vdots & \vdots & \ldots & \vdots & \vdots \\
-1 & -1 & -1 & -1 & \ldots & n-1 & -1 \\
-1 & -1 & -1 & -1 & \ldots & -1 & n-1
\end{bmatrix}
\]
It doesn't really matter which minor matrix we choose here, so let's take $M_1$ which is essentially just $Q$ but with a lower dimension by $1$; there are now $n-1$ rows and columns instead of $n$. We notice that the determinant of this matrix is precisely the characteristic polynomial of the $(n-1)\times(n-1)$ matrix with all $1$ entries. Well, such a matrix would represent a constant function that took all basis vectors to the same thing (all 1's in every component). Thus the only eigenvalue for such a transform would be $0$ and $n-1$ itself; indeed, the vector $(1,1,\ldots,1)$ will always get mapped to $(n-1)$ by this transformation. We know it's characteristic polynomial is of degree $n-1$, and has precisely two roots: $n-1$ and $0$. Thus it would have to be of the form $P(x)=(x-(n-1))x^{n-2}$, which satisfies all three conditions. But, as we see from the diagonal in the matrix we are evaluating, the determinant is actually equal to the characteristic polynomial evaluated at $n$. Thus we have $P(n)=(n-(n-1))n^{n-2}=n^{n-2}$. This is what we were trying to prove, so we are done.
\end{document}  