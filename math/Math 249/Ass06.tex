\documentclass[11pt]{article}
\usepackage{geometry}                % See geometry.pdf to learn the layout options. There are lots.
\geometry{letterpaper}                   % ... or a4paper or a5paper or ... 
%\geometry{landscape}                % Activate for for rotated page geometry
\usepackage[parfill]{parskip}    % Activate to begin paragraphs with an empty line rather than an indent
\usepackage{graphicx}
\usepackage{amssymb,amsmath}
\usepackage{fullpage}
\usepackage{epstopdf}
\DeclareGraphicsRule{.tif}{png}{.png}{`convert #1 `dirname #1`/`basename #1 .tif`.png}

\title{MATH 249 - Assignment 6}
\author{Adrian Petrescu (\#20240298)}
%\date{}                                           % Activate to display a given date or no date

\begin{document}
\maketitle
%\section{}
%\subsection{}

\textbf{1. Let $G$ be a simple planar graph that does not contain a triangle. Show that the vertices of $G$ can be properly coloured with at most 4 colours.}

The statement that $G$ contains no cycles of length 3, along with the fact that it is simple, ensures that the girth of $G$ is $\geq4$. We imitate the proof of the five-colour theorem. The planarity of $G$ ensures that it contains a vertex, $v_0$, of at most degree 5; we proceed by induction on such a vertex. The graph $G\backslash v_0$ is 4-colourable, since its size is less than $G$. Thus if we can properly insert $v_0$ back into the graph in a way that does not ruin the colouring, then our inductive step would be complete.

If $\deg{(v_0)}\leq3$, then adding this node back is trivial; we simply give it one of the four colors that remain.

Let's look at the case where $\deg{(v_0)}=4$. Let's call the neighbors of $v_0$: $v_1,v_2,v_3,v_4$. If all four colours are not used on these four vertices, then simply color $v_0$ with one of the remaining colours, and we're done. Otherwise, assume without loss of generality that $v_1$ is coloured with colour 1, $v_2$ is coloured with colour 2, and so on. Now, consider the subgraph of $G$ called $H_{13}$ which is the set of all vertices coloured either $1$ or $3$, and the edges connecting such vertices. Now, either $v_1$ and $v_3$ are in the same connected component of $H_{13}$, or they are not. If not, then simply reverse all the 1's and 3's in the connected component of $H_{13}$ containing $v_1$, and then paint $v_0$ with $1$, and we are done. Otherwise, they are in the same component. Then consider the subgraph $H_{24}$, similarly defined. By the planarity of $G$, $v_2$ and $v_4$ must be in different connected components of $H_{24}$. So switch all the 2's and 4's in the connected component of $H_{24}$ containing $v_2$, and paint $v_0$ with colour 2, and we are done this case as well.

The remaining case is where $\deg{(v_0)}=5$. Let's call the neighbours of $v_0$: $v_1, v_2, v_3, v_4, v_5$. If all four colours are not used on these five vertices, then pick one of the unused colours and paint $v_0$ with it, and we're done. Otherwise, we must necessarily have all four colours used up, and exactly one of them appearing twice. Without loss of generality, assume $v_1$ is painted with 1, $v_2$ is painted with 2, and so on, with the overlapping $v_5$ being painted with 1 also. Then consider $H_{24}$ again; once more, if $v_2$ and $v_4$ fall on different components of $H_{24}$, simply swap one of the vertices and repaint $v_0$ with the newly freed color. Otherwise, they are in the same component.

\textbf{5. Use the Matrix-Tree Theorem to show that for all $n\geq1$, the complete graph $K_n$ has $n^{n-2}$ spanning trees.}

It is clear that $K_n$ is $(n-1)$-regular, so the main diagonal will have entirely $n-1$ across it. Moreover, every vertex is connected to every other vertex by precisely on edge in $K_n$, so all other non-diagonal entries will be $-1$ in the Laplacian matrix. So we have so far
\[
Q=\begin{bmatrix}
n-1 & -1 & -1 & -1 & \ldots & -1 & -1 \\
-1 & n-1 & -1 & -1 & \ldots & -1 & -1 \\
-1 & -1 & n-1 & -1 & \ldots & -1 & -1 \\
-1 & -1 & -1 & n-1 & \ldots & -1 & -1 \\
\vdots & \vdots & \vdots & \vdots & \ldots & \vdots & \vdots \\
-1 & -1 & -1 & -1 & \ldots & n-1 & -1 \\
-1 & -1 & -1 & -1 & \ldots & -1 & n-1
\end{bmatrix}
\]
It doesn't really matter which minor matrix we choose here, so let's take $M_1$ which is essentially just $Q$ but with a lower dimension by $1$; there are now $n-1$ rows and columns instead of $n$. We notice that the determinant of this matrix is precisely the characteristic polynomial of the $(n-1)\times(n-1)$ matrix with all $1$ entries. Well, such a matrix would represent a constant function that took all basis vectors to the same thing (all 1's in every component). Thus the only eigenvalue for such a transform would be $n-1$ itself; so we know it's characteristic polynomial is of degree $n-1$, and has precisely one root: $n-1$. Thus it would have to be of the form $P(x)=(x-(n-1))x^{n-2}$, which satisfies both conditions. But, as we see from the diagonal in the matrix we are evaluating, the determinant is actually equal to the characteristic polynomial evaluated at $n$. Thus we have $P(n)=(n-(n-1))n^{n-2}=n^{n-2}$. This is what we were trying to prove, so we are done.
\end{document}  