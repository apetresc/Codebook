\documentclass[a4paper,10pt]{article}
\usepackage{amsmath}
\usepackage{fullpage}
\usepackage{amsfonts}
\newcommand{\R}{\ensuremath{\mathbb{R}}}
\newcommand{\C}{\ensuremath{\mathbb{C}}}
\newcommand{\Z}{\ensuremath{\mathbb{Z}}}
%opening
\title{MATH 145: Assignment 8}
\author{Adrian Petrescu (\#20240298)}

\begin{document}

\maketitle

\textbf{1. (a) Let $F$ be a field. Show that if a polynomial $p(x)\in F[x]$ has degree $2$ or $3$, then
$p(x)$ is irreducible in $F[x]$ if and only if $p(x)$ has no roots in $F$.}

We will show each part of the statement to be true separately. First, we must show that $p(x)$ is irreducible if it has no roots in $F$. If $p(x)$ has no roots in $F$, this means there is no element $c\in F$ such that $p(c)=0$. By Corollary 1.3, this is equivalent to saying that there is no $c$ such that the linear polynomial $(x-c)$ divides $p(x)$; however, all linear polynomials can be written in the form $a(x-c)$, so if $(x-c)$ does not divide $p$, then neither does any polynomial of degree 1. So that means for every factorization $p=qr$ in $F[x]$, neither $q$ nor $r$ can be of degree one. If we take the degree of both sides, we get $\deg{(p)}=\deg{(qr)}=\deg{(q)}+\deg{(r)}$. If $\deg{(p)}=2$, since neither $q$ or $r$ are degree $1$, one must be of degree $2$ and the other of degree $0$, which means one or the other is constant, which satisfies our definition for an irreducible polynomial. Similarly, if $\deg{(p)}=3$, one factor must be of degree $3$ and the other of degree $0$, so it is still irreducible.

Conversely, we must prove that if $p$ is irreducible, it has no roots. By the definition for irreducibility, we know that for every factorization $p=qr$ in $Z[x]$, either $q$ or $r$ is constant, that is, of degree $0$. (Without loss of generality, we say that $\deg{(q)}=0$.) So in order for the degree of the right-hand side to match the degree of $p$, $\deg{(p)}=\deg{(q)}+\deg{(r)}\implies\deg{(p)}=0+\deg{(r)}\implies\boxed{\deg{(p)}=\deg{(r)}}$. So $r$ is either quadratic or cubic (since $p$ is), not linear, thus there is never a factorization with a linear factor, so $(x-c)$ can never divide $p$, and by Corollary 1.3, this implies that there is no $c$ such that $p(c)=0$; that is, $p$ has no roots, which completes our proof.

\textbf{1. (b) Find an example of a field $F$ and a polynomial $p(x)$ of degree greater than $3$ for which statement
$(1)$ above is false.}

All we must do is find a polynomial that is reducible, but has no roots. (We cannot do it the other way around, because if $p$ has roots, then $p(c)=0$ and by Corollary 1.3, $(x-c)$ divides $p$, so it is not irreducible). However, if we take two irreducible polynomials $u,v$ of, say, degree 2, then we know that the product $p=uv$ is reducible. But since $u=ab$ is irreducible and $v=cd$ is irreducible, we can say without loss of generality that $a,c$ are constant (degree 0) and $b,d$ are quadratic (degree 2), so $p=qr=abcd$. Regardless of how we pair up the factors $a,b,c,d$ into the two factors $q,r$, they are all of degree either $0$ or $2$, so there's no way to combine them to get a factor of degree $1$; thus, no linear polynomial divides $p$, thus it is still irreducible.

So now we just have to pick an example. Take $x^2+1$ which we know to be irreducible and without roots over $\R$. Then the fourth-degree polynomial $\boxed{p(x)=\left(x^2+1\right)^2=x^4+2x^2+1}$ is reducible over $\R$ but without roots in $\R$. 

\textbf{1. (c) Is statement (1) true for polynomials of degree 1? Explain in 20 words or less.}

No, it is false for linear polynomials simply because they are all irreducible but they all have a root.

\textbf{2. Let $p(x)=x^4+1$.}

\textbf{(a) Prove that $p(x)$ is irreducible in $Q[x]$.}

By Gauss' Lemma, $p$ will factor in $Q[x]$ if and only if it factors in $Z[x]$. Additionally, since $p$ is monic, any factors it may have will also be monic. Lastly, we see that $p$ has no roots (since by the rational roots test, they would have to be of the form $\pm\frac11$, but $p(\pm1)\not=0$), so by the same reasoning as in 1(b), if $p$ were reducible, it must factor into two irreducible quadratic polynomials. So we have so far narrowed down the possible factorization of $p$ into a product of two monic quadratic polynomials in $Z[x]$. So let us assume that $p$ does indeed factor in this way, and seek a contradiction.

\begin{align*}
p(x)=&(x^2+ax+b)(x^2+cx+d)&&(a,b,c,d\in\Z)\\
=&x^4+cx^3+dx^2+ax^3+acx^2+adx+bx^2+bcx+bd\\
x^4+1=&x^4+(a+c)x^3+(b+d+ac)x^2+(ad+bc)x+bd
\end{align*}

It follows, then, that $a+c=b+d+ac=ad+bc=0$ and $bd=1$. Since these are all integers, the only solution to the latter equation is $b=d=\pm1$. Also, by the first equation, $a=-c$. Substituting both of the values, we get
\begin{align*}
b+d+ac=&0\\
\pm2-a^2=&0\\
a^2=\pm2
\end{align*}

This is a contradiction, since $\sqrt2$ is not an integer, but $a$ is. So such a factorization does not exist, and thus $p(x)$ is irreducible over $Q[x]$.

\textbf{2. (b) Factor $p(x)$ completely in $\R[x]$ and in $\C[x]$.}

Similarly to above, since $p$ is monic, any factors it may have will also be monic. Also, since $p$ has no roots, its factors in $\R$ will be irreducible, monic, quadratic polynomials. Then,
\begin{align*}
p(x)=&(x^2+ax+b)(x^2+cx+d)&&(a,b,c,d\in\R)\\
=&x^4+cx^3+dx^2+ax^3+acx^2+adx+bx^2+bcx+bd\\
x^4+1=&x^4+(a+c)x^3+(b+d+ac)x^2+(ad+bc)x+bd
\end{align*}
So we simply have to solve the system of equations in four unknowns over $\R$:
\begin{align}
a+c=&0\\
bd=&1\\
b+d+ac=&0\\
ad+bc=&0
\end{align}

By $(1)$, we know $a=-c$. Substitute this into $(4)$ to get $ad+bc=ad-ab=a(d-b)=0\implies d=b$. (It is safe to divide by $a$, since we know from $(3)$ that $a\not=0$, because if it were then $(3)$ would give us that $b=-d$ and that contradicts $(2)$). So we substitute $b=d$ into $(2)$ to get $\boxed{b=d=1}$. Lastly, we plug everything into $(3)$ to get
\begin{align*}
b+d+ac=&0\\
1+1-a^2=&0\\
a^2=&2\\
a=&\sqrt2
\end{align*}

And plugging that into $(1)$, we get $\boxed{c=-\sqrt2}$. So we have solved the system and have our factorization in $\R$:
\[p(x)=(x^2+\sqrt2x+1)(x^2-\sqrt2x+1)\]

$p(x)$ can be factored even further over $\C$. We know from Corollary 1.3 that if $p(c)=0$ then $(x-c)$ divides $p(x)$; so if we find the linear roots of $p(x)$, we can easily find the factors. Luckily, we can use the quadratic formula to find the roots of each quadratic factor in $\C$:
\begin{align*}
x=&\frac{-b\pm\sqrt{b^2-4ac}}{2a}\\
=&\frac{\pm\sqrt2\pm\sqrt{2-4}}{2}\\
=&\frac{\pm\sqrt2\pm\sqrt{2}i}{2}
\end{align*}

So the four roots of $p(x)$ are simply all the possible combinations of $\pm$ in the above equation; that is, \[\frac{\sqrt2+\sqrt2i}{2},\frac{\sqrt2-\sqrt2i}{2},\frac{-\sqrt2+\sqrt2i}{2},\frac{-\sqrt2-\sqrt2i}{2}\] And thus the complete factorization of $p(x)$ in $\C$ is:
\[p(x)=\left(x-\frac{\sqrt2+\sqrt2i}{2}\right)\left(x-\frac{\sqrt2-\sqrt2i}{2}\right)\left(x-\frac{-\sqrt2+\sqrt2i}{2}\right)\left(x-\frac{-\sqrt2-\sqrt2i}{2}\right)\]

\textbf{2. (c) Factor $p(x)$ completely in each of the following domains:}

\textbf{           $\text{       }$ i. $\Z_2[x]$}

By the rational roots test, we see that any rational roots must be in the form $\pm1$. Right away, we see that in $\Z_2[x]$, $p(1)=0$, so $1$ is a root, so $(x-1)$ is a factor. Performing synthetic division (reducing modulo 2 at every step) on $p$ by $(x-1)$, we get $p(x)=(x-1)(x^3+x^2+x+1)$. Once again, the cubic factor has a root of $1$, so there is another factor of $(x-1)$. Performing synthetic division modulo 2 again, we get $p(x)=(x-1)^2(x^2+1)$. At this point, we can easily factor $(x^2+1)$ modulo 2, since $1\equiv-1\pmod{2}$ so $(x^2+1)\equiv(x^2-1)$ which is a difference of squares and factors to $(x+1)(x-1)$; and just for simplicity, we can turn $(x+1)$ into $(x-1)$ (again, because $1\equiv-1\pmod{2}$), so our final factorization is \[x^4+1\equiv(x+1)^4\pmod{2}\]

\textbf{           $\text{       }$ ii. $\Z_5[x]$}

The rational roots test tells us that $p(x)$ has no roots in $\Z_5[x]$; therefore we know it is the product of two monic irreducible quadratic polynomials. That is, $x^4+1=(x^2+ax+b)(x^2+cx+d)$. We know that the constant term, $bd$, must be congruent to $1$ modulo $5$; this is doable as $1\cdot1$,$2\cdot3$, or $4\cdot4$. This lowers the amount of polynomials we have to look at. Additionally, we know that (since the coefficient on $x^2$ is $0$), that $a+c\equiv0\pmod{5}$. (Note that we are essentially trying to solve the system of equations from part 2 (b), but modulo 5). So with these restrictions in place, we can begin to list quadratic polynomials. At each one, we check if it is irreducible (by the factor theorem). If it is, then we try to divide it into $x^4+1$. We begin with the smallest values for $a,b,c,d$ that fit the restrictions that we derived; that is, $a,c=0$ and $b,d=1$.
\setcounter{equation}{0}
\begin{align}
x^2+1 &&\text{(Reducible; $(x-3)$ is a factor, so we ignore this.)}\\
x^2+2 &&\text{(Irreducible; $0+2=2, 1+2=3, 4+2=6\equiv1, 9+2=11\equiv1, 16+2=18\equiv3$)}
\end{align}

We've found an irreducible polynomial that fits our restrictions, so we check to see if it divides $p(x)$. We perform the neccesary long division to divide $p(x)$ by $(x^2+2)$ (given in the margin), and we discover that it does indeed divide cleanly. So the quotient, $(x^2+3)$ is the other factor, and it seems we were lucky enough to hit upon these very early in our listing of the irreducible quadratic polynomials of $Z_5[x]$. So our factorization is \[x^4+1\equiv(x^2+2)(x^2+3)\pmod{5}\]

\textbf{           $\text{       }$ iii. $\Z_7[x]$}

Once again by the rational roots test, any potential roots must be of the form $\pm1$, but $p(\pm1)\not=0$ in $\Z_7$, so $p$ has no roots, and thus if it factors, it must be into two monic irreducible quadratic polynomials. So $p(x)=(x^2+ax+b)(x^2+cx+d)$, and by the same procedure as in 2(b), we arrive at \[x^4+1=x^4+(a+c)x^3+(b+d+ac)x^2+(ad+bc)x+bd\] And so we must solve the system of congruences
\setcounter{equation}{0}
\begin{align}
a+c\equiv&0\pmod7\\
bd\equiv&1\pmod7\\
b+d+ac\equiv&0\pmod7\\
ad+bc\equiv&0\pmod7
\end{align}

From (1) we immediately get that $a\equiv-c\pmod{7}$. Substituting this into (4), we get $(ab-ad)\equiv0\pmod{7}$. We can divide both sides by $0$ since $7$ is prime and $a<7$ so it must be relatively prime to $7$; thus $b-d\equiv0\pmod7\implies b\equiv d\pmod7$. Substituting this new fact into (2), we get $b^2\equiv1\pmod7$, which by trial and error tells us that $b$ and $d$ are congruent to $\pm1$ modulo 7. Finally, we substitute everything we know into (3) to get
\begin{align*}
2d-a^2\equiv&0\pmod7\\
2d\equiv&a^2\pmod7\\
\end{align*}

Right away we can tell that $d\equiv1\pmod7$ because no perfect square is congruent to $-1$ modulo 7 (by trial and error). So substituting $d\equiv1\pmod7$ we get
\begin{align*}
a^2\equiv2\pmod7
\end{align*}
By trial and error, we see that $a=3$ and $a=4$ both satisfy this congruence. So $b\equiv d\equiv1\pmod7$ and $a\equiv-c\equiv3,4\pmod7$. So we have our two polynomials simply by substituting in $a,b,c,d$:
\[x^4+1\equiv(x^2+3x+1)(x^2+4x+1)\pmod7\]

\textbf{3. Let $f(x)=x^5+3x^4+2x^3+x^2+x-2$ and $g(x)=x^5+2x^4+2x^3-x^2-4x+2$. Use the Euclidean
algorithm to find $\gcd{(f, g)}$ and show how to express a primitive associate of $\gcd{(f, g)}$ as a polynomial
linear combination of $f$ and $g$.}

The technicalities of the Euclidian algorithm with all the dirty charts and calculations are given on the attached piece of graph paper, because it is just way too ugly for \LaTeX. Suffice to say that the chart shows that $r_3=0$, which implies that $\gcd{(f,g)}=r_2=-10x^2-10x+10$. However, this is not a primitive polynomial, so we take out a factor of $-\frac1{10}$, to obtain $\boxed{\gcd{(f,g)}=x^2+x-1}$.

Lastly, we know that \[r_2=s_2\cdot f(x)+t_2 g(x)\]
So then,
\begin{align*}
\frac{-r_2}{10}=&-\frac{s_2f(x)}{10}-\frac{t_2g(x)}{10}\\
\gcd{(f,g)}=&\frac{(x+2)}{10}f(x)+\frac{-(1+x+2)}{10}g(x)
\end{align*}

\textbf{4. (a) Find a primitive polynomial $p(x)$ in $Z[x]$ of degree $6$ having $\sqrt3 + \sqrt[3]5$ as a root.}

We know that the root, $x$, is: $x=\sqrt[3]5+\sqrt3$. Manipulating this algebraically, we see that
\begin{align*}
x-\sqrt3=&\sqrt[3]5\\
(x-\sqrt3)^3=&5\\
x^3-3\sqrt3x^2+9x-3\sqrt3-5=&0\\
x^3+9x-5=&3\sqrt3x^2+3\sqrt3\\
(x^3+9x-5)^2=&(3\sqrt3x^2+3\sqrt3)^2\\
x^6+18x^4-10x^3+81x^2-90x+25=&27x^4+54x^2+27\\
x^6-9x^4-10x^3+27x^2-90x-2=&0
\end{align*}

The polynomial we arrive at is clearly primitive, since it is monic, and all its coefficients are integers. A quick numerical check in Maple confirms that this polynomial does indeed have $\sqrt3+\sqrt[3]5$ as a root, so we are done.

\textbf{4. (b) Show that $p(x)$ factors in $\Z_3[x]$ as $(x+1)^6$ and in $\Z_5[x]$ as $(x^2+2)^3$.}

First, we reduce $p(x)$ from part (a) modulo $3$ by reducing each individual coefficient:
\[p(x)\equiv(x^6-x^3-2)\equiv(x^6+2x^3+1)\pmod{3}\]

Now we simply expand $(x+1)^6$, modulo 3:

\begin{align*}
(x+1)^6=\left((x+1)^2\right)^3=&(x^2+2x+1)^3\\
=&(x^2+2x+1)(x^2+2x+1)(x^2+2x+1)\\
=&(x^4+2x^3+x^2+2x^3+4x^2+2x+x^2+2x+1)(x^2+2x+1)\\
=&(x^4+x^3+x+1)(x^2+2x+1)\\
=&(x^6+2x^5+x^4+x^5+2x^4+x^3+x^3+2x^2+x+x^2+2x+1)\\
=&(x^6+3x^5+3x^4+2x^3+3x^2+3x+1)\\
=&x^6+2x^3+1\\
\end{align*}

This matches our statement for $p(x)$, so they are equivalent; thus $p(x)\equiv(x+1)^6\pmod{3}$.

We proceed in a very similar fashion for the factorization in $\Z_5[x]$; first we reduce $p$ modulo $5$:
\[p(x)\equiv(x^6-4x^4+2x^2-2)\equiv(x^6+x^4+2x^2+3)\pmod{5}\]

And then we expand $(x^2+2)^3$ modulo 5:
\begin{align*}
(x^2+2)^3=&(x^2+2)^2(x^2+2)\\
=&(x^4+4x^2+4)(x^2+2)\\
=&(x^6+2x^4+4x^4+8x^2+4x^2+8)\\
=&x^6+6x^4+12x^2+8\\
=&x^6+x^4+2x^2+3
\end{align*}

Again, this matches our reduced form of $p(x)$, so they are equivalent, and thus $p(x)\equiv(x^2+2)^3\pmod{5}$.

\textbf{BONUS: Prove or disprove Nadia's Conjecture; that is, if
\[f(x)=\frac{a_0}{b_0}+\frac{a_1}{b_1}x+\frac{a_2}{b_2}x^2+...+\frac{a_n}{b_n}x^n\]
is a nonzero polynomial in $\mathbb{Q}[x]$, where each $\frac{a_i}{b_i}$ is a fraction in lowest terms, and if $a=\gcd{(a_0,a_1,...,a_n)}$, and $b=\text{lcm}(b_0,b_1,...,b_n)$, then the content of $f$ is $\frac{a}{b}$. (i.e., $f(x)=\frac{a}{b}f_1(x)$ for some primitive polynomial $f_1(x)$.)} 

Suppose $b_0=b_1=b_2=...=b_n=1$; in other words, suppose $f(x)\in\Z[x]$. Then $f(x)=a_0+a_1x+a_2x^2+...+a_nx^n$. If $a=\gcd{(a_0,a_1,a_2,...,a_n)}$, then let $f_1(x)=\frac{f(x)}{a}=\frac{a_0}{a}+\frac{a_1}{a}x+...+\frac{a_n}{a}$. Clearly $f_1(x)$ is a primitive polynomial since all the coefficients must be relatively prime to each other, since we are dividing by any common factors they may have had between them. Thus $f_1(x)=\frac{f(x)}{a}\implies f(x)=\frac{a}{1}f_1(x)$. This is what we were trying to prove for the case of all the $b$'s being equal to $1$.

So we have shown that Nadia's Conjecture holds for polynomials in the integers. We will call this "Nadia's Little Conjecture". Now let us get rid of our earlier assumption that $b_0=b_1=b_2=...b_n=1$; now they are all just some integers. Then if $b=\text{lcm}(b_0,b_1,b_2,...,b_n)$, let $f_2(x)=bf(x)$. Clearly by multiplying every coefficient by a number $b$ which is a multiple of its denominator, we have cancelled out all the denominators of the coefficients, so $f_2(x)\in\Z[x]$. Therefore, Nadia's Little Conjecture holds, and says that $f_2(x)=\frac{a}{1}f_1(x)$, where $f_1$ is a primitive polynomial. So then $bf(x)=\frac{a}{1}f_1{x}\implies f(x)=\frac{a}{b}f_1(x)$, which is what we were trying to prove. Thus Nadia's Conjecture is now Nadia's Theorem over the rationals.

\end{document}
