\documentclass[11pt]{amsart}
\usepackage{geometry}                % See geometry.pdf to learn the layout options. There are lots.
\geometry{letterpaper}                   % ... or a4paper or a5paper or ... 
%\geometry{landscape}                % Activate for for rotated page geometry
\usepackage[parfill]{parskip}    % Activate to begin paragraphs with an empty line rather than an indent
\usepackage{graphicx}
\usepackage{amssymb}
\usepackage{epstopdf}
\usepackage[usenames,dvipsnames]{color}
\usepackage{fullpage}
\usepackage{listings}
\DeclareGraphicsRule{.tif}{png}{.png}{`convert #1 `dirname #1`/`basename #1 .tif`.png}

\definecolor{gray}{cmyk}{0.06,0.04,0.03,0}

\title{CS 371 - Assignment 1}
\author{Adrian Petrescu (\#20240298)}
%\date{}                                           % Activate to display a given date or no date

\begin{document}
\maketitle
%\section{}
%\subsection{}

\textbf{1. What is the largest number you can add to 1 and have the answer still be 1?}

Consider the following Python code (in Python, 'float' is the IEEE double-precision floating point representation):
\\
\lstset{language=Python,showstringspaces=false,numbers=left,frame=single,backgroundcolor=\color{gray}}

\lstinputlisting{epsilon.py}

The output when this code is run is:
\begin{verbatim}
eps = 1.11022302463e-16
1.0 + eps = 1.0
\end{verbatim}
Thus the machine epsilon is around $1.11022\times 10^{-16}$.

\textbf{2. What is the result in your computer of the operation of dividing by zero?}

As most programming languages do, Python will throw an exception upon attempting to divide by zero. The following code:

\lstinputlisting{divzero.py}

will result in:
\begin{verbatim}
Traceback (most recent call last):
  File "<stdin>", line 1, in <module>
ZeroDivisionError: float division
\end{verbatim}

However, consider a more ancient language such as the following program written in C:

\lstset{language=C}
\lstinputlisting{divzero.c}

This will successfully compile and execute, returning the exceptional float number
\begin{verbatim}
result = inf
\end{verbatim}

\textbf{3. What does your computer do if you compute $\sqrt{-1}$? }

This is highly language-specific, since it depends on the implementation of the square root function. For instance, the Python interpreter gives
\begin{verbatim}
>>> sqrt(-1)
nan
\end{verbatim}
while the following C program:
\lstinputlisting{sqrt.c}
gives:
\begin{verbatim}
sqrt(-1) = -2147483648
\end{verbatim}
On the other hand, a more enlightened language will behave quite differently:
\begin{verbatim}
Welcome to DrScheme, version 4.0 [cgc].
Language: Pretty Big (includes MrEd and Advanced Student).
> (sqrt -1)
0+1i
\end{verbatim}

\textbf{4. If your program did not crash in \#3, what is the result of the logical test $num == num$?}

In C, the augmented program
\lstinputlisting{sqrteq.c}
returns
\begin{verbatim}
(-2147483648 == -2147483648): 1
\end{verbatim}
In Python, NaN is never equal to NaN, and in the ever-enlightened Scheme, \texttt{(eq 0+1i 0+1i)} always evaluates to \texttt{true}.

\textbf{5. Let�s do some algebra. Check if the given statements are true and report the results.}

It is not enough to check the algebra for just one set of three numbers $x,y,z$. It is conceivable that some forms of rounding will occur with one set of numbers, but not another, which would skew the results. Therefore, we run the four tests on 500 sets of random doubles:
\lstset{language=Python}
\lstinputlisting{algebra.py}

which returns:
\begin{verbatim}
Commutativity: True
Distributativity: False
Cancellation: False
Square root: True
\end{verbatim}

\textbf{6. What is the result of the given operation?}

Python has unbounded integers, so we will do this in plain C instead:
\lstset{language=C}
\lstinputlisting{overflow.c}

This, of course, prints
\begin{verbatim}x + y = -1294967296\end{verbatim}
because $x + y = 3000000000$ which is outside the range of a 32-bit integer, so the value wraps around.
\end{document}  