\documentclass[11pt]{amsart}
\usepackage{geometry}                % See geometry.pdf to learn the layout options. There are lots.
\geometry{letterpaper}                   % ... or a4paper or a5paper or ... 
%\geometry{landscape}                % Activate for for rotated page geometry
\usepackage[parfill]{parskip}    % Activate to begin paragraphs with an empty line rather than an indent
\usepackage{graphicx}
\usepackage{amssymb}
\usepackage{epstopdf}
\usepackage{fullpage}
\usepackage{listings}

\DeclareGraphicsRule{.tif}{png}{.png}{`convert #1 `dirname #1`/`basename #1 .tif`.png}

\title{CS 341 - Assignment 2}
\author{Adrian Petrescu (\#20240298)}
%\date{}                                           % Activate to display a given date or no date

\begin{document}
\maketitle
%\section{}
%\subsection{}

\textbf{1. Write a RAM program for the following problem: you are given a list $L$ of $n-1$ integers $x_1,x_2,\ldots,x_{n-1}$ for some $n\geq2$, and you are told that $L$ contains every number in $\{1,2,\ldots,n\}$ exactly once (in unknown order), except that one number $m$ is missing. You must find $m$, the missing number. \\\\
Try to find the most efficient algorithm (in terms of time and space) you can, under the unit-cost model. Your solution will be marked on efficiency and correctness.}

We note that the sum of the entire set $\{1,2,\ldots,n\}$ is equal to $\frac{n(n+1)}{2}$, a well-known identity. Since $L$ is missing exactly one of the elements of that set, $m$, we know that the sum of $L$ will be 
\[\sum{L}=\frac{n(n+1)}{2}-m \implies \boxed{m=\frac{n(n+1)}{2}-\sum{L}}\]
Thus our algorithm need only sum together the elements of $L$ (in $O(n)$ time), and subtract it from $\frac{n(n+1)}{2}$, in $O(1)$ time, for a total runtime cost of $O(n)$. Since it is absolutely necessary to read in every element of $L$ in order to solve this problem, we can be reasonably certain that no substantial improvement over this algorithm is possible.

So let us get down to actually implementing the algorithm:

\lstset{showstringspaces=false,numbers=left,frame=single}
\lstinputlisting{q1.ram}

$\quad$\\

\textbf{2. Analyze your algorithm's worst-case time and space complexity using the unit-cost method. Express your answer as accurately as possible, in terms of $n$ and (possibly) $m$.}

The algorithm consists of a prefix, a loop that is repeated $n-1$ times (once for each element in $L$), and a suffix. The length of the prefix is $6$, the length of the suffix is $4$, and the loop has length $7$. Thus the unit-cost runtime is $7(n-1)+10$.

Moreover, the algorithm makes use only of registers $v_0,v_1,v_2,v_3$ no matter how large the input is. Since the unit-cost model charges only $1$ unit of space for each register, this gives the algorithm a constant space cost of $4$.

\textbf{3. Analyze your algorithm's worst-case time and space complexity using the log-cost method. You may express your answer using asymptotic notation.}



\end{document}  
