\documentclass[11pt]{amsart}
\usepackage{geometry}                % See geometry.pdf to learn the layout options. There are lots.
\geometry{letterpaper}                   % ... or a4paper or a5paper or ... 
%\geometry{landscape}                % Activate for for rotated page geometry
\usepackage[parfill]{parskip}    % Activate to begin paragraphs with an empty line rather than an indent
\usepackage{graphicx}
\usepackage{amssymb}
\usepackage{epstopdf}
\usepackage{fullpage}
\usepackage{listings}

\DeclareGraphicsRule{.tif}{png}{.png}{`convert #1 `dirname #1`/`basename #1 .tif`.png}

\title{CS 341 - Assignment 2}
\author{Adrian Petrescu (\#20240298)}
%\date{}                                           % Activate to display a given date or no date

\begin{document}
\maketitle
%\section{}
%\subsection{}

\textbf{1. Write a RAM program for the following problem: you are given a list $L$ of $n-1$ integers $x_1,x_2,\ldots,x_{n-1}$ for some $n\geq2$, and you are told that $L$ contains every number in $\{1,2,\ldots,n\}$ exactly once (in unknown order), except that one number $m$ is missing. You must find $m$, the missing number. \\\\
Try to find the most efficient algorithm (in terms of time and space) you can, under the unit-cost model. Your solution will be marked on efficiency and correctness.}

We note that the sum of the entire set $\{1,2,\ldots,n\}$ is equal to $\frac{n(n+1)}{2}$, a well-known identity. Since $L$ is missing exactly one of the elements of that set, $m$, we know that the sum of $L$ will be 
\[\sum{L}=\frac{n(n+1)}{2}-m \implies \boxed{m=\frac{n(n+1)}{2}-\sum{L}}\]
Thus our algorithm need only sum together the elements of $L$ (in $O(n)$ time), and subtract it from $\frac{n(n+1)}{2}$, in $O(1)$ time, for a total runtime cost of $O(n)$. Since it is absolutely necessary to read in every element of $L$ in order to solve this problem, we can be reasonably certain that no substantial improvement over this algorithm is possible.

So let us get down to actually implementing the algorithm:
\lstset{showstringspaces=false,numbers=left,frame=single}

\lstinputlisting{q1.ram}

$\quad$\\

\textbf{2. Analyze your algorithm's worst-case time and space complexity using the unit-cost method. Express your answer as accurately as possible, in terms of $n$ and (possibly) $m$.}

The algorithm consists of a prefix, a loop that is repeated $n-1$ times (once for each element in $L$), and a suffix. The length of the prefix is $6$, the length of the suffix is $4$, and the loop has length $7$. Thus the unit-cost runtime is $7(n-1)+10$.

Moreover, the algorithm makes use only of registers $v_0,v_1,v_2,v_3$ no matter how large the input is. Since the unit-cost model charges only $1$ unit of space for each register, this gives the algorithm a constant space cost of $4$.

\textbf{3. Analyze your algorithm's worst-case time and space complexity using the log-cost method. You may express your answer using asymptotic notation.}

We can see that the algorithm performs exactly the same number of steps and uses exactly the same number of registers across all inputs of size $n$; the only difference is the size of the numbers being handled. Obviously the worst case is when $m=1$ is missing since this maximizes $\sum{L}$.

Consider the run-time of the prefix. Since the first symbol of the input tape is simply $n$, and is therefore independent of $m$, the prefix (which simply computes $n(n+1)/2$ and stores it in $r2$) will always have the same runtime across all inputs of size $n$:
\begin{align*}
T_{\text{prefix}}(n,m)=&\underbrace{(l(n)+l(1))}_{\texttt{READ 1}}+\underbrace{l(1)}_{\texttt{LOAD 1}} +\underbrace{(l(n) + l(1))}_{\texttt{ADD =1}} + \underbrace{(l(n+1) + l(n) + l(1))}_{\texttt{MULT 1}} +  \\
& + \underbrace{(l(n(n+1)) + l(2))}_{\texttt{DIV =2}} + \underbrace{l(n(n+1)/2) + l(2)}_{\texttt{STORE 2}}
\end{align*}
Since $l(n(n+1))=l(n^2+n)=\log_2{(n^2+n)}=O(\log_2{(n^2)})=O(2\log_2{(n)})=O(\log_2{n})$, and $l(n(n+1))$ is the dominant term in $T_{\text{prefix}}(n)$, we can say that the prefix time is in $O(\log_2{n})$.

Similarly, the runtime of the suffix is
\begin{align*}
T_{\text{suffix}}(n,m)=&\underbrace{(l(2)+l(n(n+1)/2))}_{\texttt{LOAD 2}} + \underbrace{(l(n(n+1)/2) + l((n(n+1)/2)-m))}_{\texttt{SUB 3}} + \\
& + \underbrace{(l(0) + l(m))}_{\texttt{WRITE 0}}+\underbrace{1}_{\texttt{HALT}}
\end{align*}
Since the worst case is when $m=1$, by an almost identical argument to the above, the runtime $T_{\text{suffix}}(n,m)$ is in $O(\log_2{n})$.

It remains to analyze the main loop, which is executed $n-1$ times regardless of what $m$ is. We proceed as before:
\begin{align*}
T_{\text{loop}}(n,m)=&\underbrace{(l(n) + l(0))}_{\texttt{READ 0}} + \underbrace{l(n) + l(3) + l\left(\sum{L}-n\right)}_{\texttt{ADD 3}} + \underbrace{l\left(\sum L\right) + l(3)}_{\texttt{STORE 3}} \\
& + \underbrace{(l(n)+l(1))}_{\texttt{LOAD 1}} + \underbrace{(l(n)+l(1))}_{\texttt{SUB =1}} + \underbrace{l(n) + l(1)}_{\texttt{STORE 1}} + \underbrace{l(n)}_{\texttt{JUMP>0 loop}}
\end{align*}

The dominant term here is $l\left(\sum L\right)$, which as we saw before is $O(\log_2{n})$. Thus the total runtime of the algorithm in the log-cost model is
\begin{align*}
T(n,m) =& T_{\text{prefix}}(n,m) + (n-1)T_{\text{loop}}(n,m)+ T_{\text{suffix}}(n,m) \\
=& O(\log_2{n}) + (n-1)O(\log_2{n}) + O(\log_2{n}) \\
=& O(n\log_2{n})
\end{align*}

So under the log-cost model, the total runtime is actually $n\log{n}$, which reflects the fact that larger numbers take longer to work with.

Space is easier to analyze. The largest value ever stored in $r_0$ is $n(n+1)$. The largest value ever stored in $r_1$ is $n$, the largest value stored in $r_2$ is $n(n+1)/2$ and the largest value ever stored in $r_3$ is $\sum L=(n(n+1)/2-m)$. Since the total space complexity is the sum of the maximum contents of each register, we have
\[S(n,m) = l(n(n+1)) + l(n) + l(n(n+1)/2) + l(n(n+1)/2 -m)\]
Each one of these terms is $O(\log_2{n})$, and there's always only $4$ registers used regardless of the size of the input, so we conclude that the space efficiency of the algorithm is $O(\log_2{n})$.
\end{document} 