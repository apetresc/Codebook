%%This is a very basic article template.
%%There is just one section and two subsections.
\documentclass[a4]{article}

\usepackage{fullpage}
\usepackage[parfill]{parskip}
\usepackage{amsmath}
\usepackage{amssymb}

\title{PMATH 345 - Assignment 1}
\author{Adrian Petrescu (\#20240298)}

\begin{document}

\maketitle

\textbf{1. (a) Let $R = \mathbb{R}[x, y]$ be the set of all polynomials in two
variables with real coefficients. Let $S$ be the subset of polynomials $p(x, y)$ such that
$p_x(0, 0) = 0$ ($p_x$ denotes the partial derivative of $p$ with respect to
$x$). Is $S$ a subring of $R$? Prove your answer.}

To check if $S$ is a subring, three things need to be checked. 

The additive identity for $R$ is the zero polynomial $p(x,y)=0$, whose partial
derivative with respect to $x$ is also the zero polynomial, which is $0$ at the origin,
and therefore in $S$. Similarly, the multiplicative inverse $p(x,y)=1$ has the
same partial derivative with respect to $x$, so it is also in $S$.

Next, we check that $S$ is closed under subtraction. Let $p(x,y),q(x,y)\in S$.
We want to check if $r(x,y)=p(x,y)-q(x,y)$ is also in $S$. It is obvious that
$r$ is a polynomial, but it remains to show that $r_x(0,0)=0$. Well,
polynomials are smooth, so we can safely use the subtraction rule for partial
derivatives. That is,
\begin{align*}
	&\frac{\partial{r}}{\partial{x}}=\frac{\partial}{\partial{x}}(p-q)=\frac{\partial p}{\partial x} - \frac{\partial q}{\partial x} \\
	\implies&r_x(x,y)=p_x(x,y)-q_x(x,y) \\
	\implies&r_x(0,0)=p_x(0,0)-q_x(0,0)=0 \\
	\implies&r\in S
\end{align*}

Lastly, we check that $S$ is closed under multiplication. Let $s(x,y)=p(x,y)q(x,y)$. Once again we can use the product rule 
for partial derivatives, so
\begin{align*}
	&\frac{\partial{r}}{\partial{x}} = q\frac{\partial p}{\partial{x}} +
	p\frac{\partial{q}}{\partial{x}} \\
	\implies &s_x(x,y) = p_x(x,y)q(x,y) + p(x,y)q_x(x,y) \\
	\implies &s_x(0,0) = 0 \\
	\implies &s\in S
\end{align*}

Thus, by the subring theorem, $S$ is a subring of $R$.

\textbf{(b) Let $R = \mathbb C$, the set of complex numbers, and let $S$ be the
subset of complex numbers $z$ of the form $z = a + b\sqrt2$, where $a, b \in
\mathbb Z$ are integers. Is $S$ a subring of $R$? Prove your answer.}

Once more, we check the three facts. The multiplicative and additive identities
for $R$ are $1+0i=1+0\sqrt2\in S$ and $0+0i=0+0\sqrt2\in S$ respectively, so
the first condition is satisfied.

Next we check that $S$ is closed under subtraction:
\begin{align*}
	(a+b\sqrt2)-(c+d\sqrt2)=(a-c)+(b-d)\sqrt2\in S \quad\text{(Since the integers
	are a ring.)}
\end{align*}
Lastly, we check closure under multiplication:
\begin{align*}
	(a+b\sqrt2)(c+d\sqrt2)=&ac+ad\sqrt2+bc\sqrt2+2bd \\
	=&(ac+2bd) + (ad+bc)\sqrt2\in S
\end{align*}
So by the subring theorem, $S$ is a subring.

\textbf{2. (a) Is the mapping from $\mathbb Z_{10}$ to $\mathbb Z_{10}$ given by
$x\to2x$ a ring homomorphism? Prove your answer.}

This mapping fails the simplest condition for a homorphism. Simply, the
multiplicative identity of $\mathbb Z_{10}$ is $1$, which should be preserved
under a homomorphism, but under this mapping, it goes to $2$ which is not the
identity for $\mathbb Z_{10}$. Indeed, any homomorphism from a ring to itself
would have to have a fixed point at $1$ since the identity is unique, but the
only element with $x=2x$ is $x=0$, so the only ring under which this could be a
homomorphism would have to have $0$ as both the additive and multiplicative
identities: that is, the zero ring.

\textbf{(b) Let $\phi:\mathbb C[x]\to\mathbb C[x]$ be defined by
$\phi(p(x)) = p(x^2)$. Is $\phi$ a homomorphism? Prove your answer.}

We check the three properties of a homomorphism. First, the $0$ and $1$ of
$\mathbb C[x]$ are $p(x)=0$ and $p(x)=1$ respectively, neither of which are
affected by $\phi$, since they are constant polynomials, so it is true that
$\phi(1)=1$.

Next, consider $p(x)=\sum_{i=0}^n{a_ix^i}$ and $q(x)=\sum_{j=0}^m{b_jx^j}$.
Then we have:
\begin{align*}
&r(x)=p(x)+q(x)
\implies \phi(r)=r(x^2)=p(x^2)+q(x^2)=\phi(p)+\phi(q)
\end{align*}
Similarly,
\begin{align*}
s(x)=p(x)q(x)\implies\phi(s)=p(x^2)q(x^2)=\phi(p)\phi(q)
\end{align*}
So it satisfies the rules for addition and multiplication, and therefore is a
homomorphism.

\textbf{3. Let $\phi:\mathbb Z_9\to\mathbb Z_3$ be a homomorphism. What is
$\phi(4)$?}

By the definition of a homomorphism, $\phi(1)=1$. Also by the definition of a
homomorphism, $\phi(p_1+p_2)=\phi{(p_1)}+\phi{(p_2)}$. Putting these two facts
together, we get
\begin{align*}
\phi(4)=&\phi(1)+\phi(1)+\phi(1)+\phi(1) \\
	=& [1]_3+[1]_3+[1]_3+[1]_3 \\
	=& [4]_3=1
\end{align*}

\textbf{4. Let $R$ be a commutative ring, and let $\phi:R\to R$ be a
homomorphism. Define the following subset of $R$: \[ R^\phi=\{x\in
R\mid\phi(x)=x\} \] Is $R^\phi$ a subring of $R$? If so, prove your answer. If
not, give a specific example of a ring $R$ and a homomorphism $\phi:R\to R$
such that $R^\phi$ is not a ring.}

Firstly, it is clear (by definition) that $\phi(1)=1\implies 1\in R^\phi$. It
also follows easily from the fact that $\phi(0\cdot a)=\phi(0)\phi(a)=\phi(0)$
for any $a\in R$, that $\phi(0)=0\implies 0\in R^\phi$. So it only remains to
show that $R^\phi$ is closed under subtraction and multiplication.

Take $r_1,r_2\in R^\phi$. We want to see if $r_1r_2$ is in $R^\phi$; well,
$\phi(r_1r_2)=\phi(r_1)\phi(r_2)=r_1r_2$, so it is in $R^\phi$.

To show that $R^\phi$ is also closed under subtraction, we first show that
$-1\in R^\phi$. We know that $\phi(1+(-1))=\phi(1)+\phi(-1)$, but since we
already have that $\phi(1)=1$ and $\phi(0)=0$, it follows that $1+\phi(-1)=0$
which implies $\phi(-1)=-1$, so $(-1)\in R^\phi$. Now, using the result of
$5(a)$, we simply have
\begin{align*}
	\phi(r_1-r_2)=&\phi(r_1+(-1)(r_2))=\phi(r_1)+\phi(-1)\phi(r_2)\\
	=&r_1+(-1)r_2=r_1-r_2\implies(r_1-r_2)\in R^\phi
\end{align*}
Thus, $R^\phi$ is a subring of $R$.

\textbf{5. Let $R$ be a commutative ring.}

\textbf{(a) Prove that $-x=(-1)x$ for all $x\in R$. (Recall that $-x$ denotes
the additive inverse of $x$, and $(-1)x$ denotes the product of $x$ with $-1$,
the additive inverse of $1$.)}

We wish to prove that $x+(-1)x=0$, showing that the additive inverse of $x$ is
simply the product of $x$ with the additive inverse of $1$. By the definition
of $1$, the left-hand side of the above is equivalent to $(1)x+(-1)x$. Then,
using the 9th property of a commutative ring, we can say
\[
(1)x+(-1)x=(1+(-1))x=0x=0
\]
Thus, $-x$ is indeed the additive inverse.

\textbf{(b) Let $\phi:R\to R$ be defined by $\phi(x) = -x$. Prove that $\phi$ is
a homomorphism if and only if $1 + 1 = 0$ in $R$.}

Suppose $\phi$ is a homomorphism. Then, by definition of a homomorphism,
$\phi(1)=1$. But, by the definition of $\phi$, $\phi(1)=-1$. In order for this
to be possible, $-1$ must be the same as $1$; so $1=-1$. Adding $1$ to both
sides we get $1+1=-1+1=0$, so the condition is necessary.

To show it is sufficient, we assume $1+1=0$ and show that the definition of a
homomorphism is met. Firstly, it is clear that
$\phi(r_1+r_2)=-(r_1+r_2)=(-1)(r_1+r_2)=-r_1-r_2=\phi(r_1)+\phi(r_2)$.
Similarly, $\phi(r_1r_2)=-r_1r_2=(-1)r_1(1)r_2$. But, since $1+1=0$, we we can
subtract $1$ from both sides to get $1=-1$, so the previous expression is
equivalent to $(-1)r_1(-1)r_2=\phi(r_1)\phi(r_2)$, so $\phi$ is also closed
under multiplication. Lastly, we need to show that $\phi(1)=1$, but as we just
stated, $1=-1$, so $\phi(1)=-1=1$, and we are done.

\textbf{6. Find all the ideals of the ring $\mathbb Z/8\mathbb Z$.}

All of the ideals of the rings of integers modulo $n$ are generated by the
elements of the ring. Since $n=8$ in this case, there's not too many to
explicitly list them all, so we will do so.

First, the rings generated by the elements $0$ and $1$ are the so-called
trivial ``zero'' ideal and ``unit'' ideal, which consist of $\{0\}$ and
$\{0,1,2,3,4,5,6,7\}$ respectively. It simply remains to list the other ideals:
\begin{itemize}
\item $2\mathbb Z_8 = \{0,2,4,6\}$
\item $3\mathbb Z_8 = \{0,3,6\}$
\item $4\mathbb Z_8 = \{0,4\}$
\item $5\mathbb Z_8 = \{0,5\}$
\item $6\mathbb Z_8 = \{0,6\}$
\item $7\mathbb Z_8 = \{0,7\}$
\end{itemize}

We can see that these are all of the ideals using the following argument. Since
the integers are well-ordered, we can sort the elements of an ideal, $I$, into
ascending order. The first such element, therefore, will always be $0$ by the
definition of an ideal. We will prove that, for the \textit{second-}lowest
element in $I$, called $\alpha_1$, the ideal generated by $\alpha_1$ is the
only ideal containing $\alpha_1$ as its second-lowest element. Since there are
only $7$ choices for the second-lowest element, the $6$ ideals listed above
(plus the unit and zero ideal) make up all $8$ ideals of $\mathbb Z_8$.

Let $\alpha_1>1\in\mathbb Z_8$ be the second-lowest element of the ideal
$\alpha_1\mathbb Z_8$ (the ideal generated by $\alpha_1$). We know that this is
the smallest ideal containing $\alpha_1$; we will show that it is also the
largest ideal containing $\alpha_1$ as its second-lowest element. Suppose, by
contradiction, that there existed some $\gamma\in I$ which was not already in
$\alpha_1\mathbb Z_8$. Then, that means that it is not of the form $n\alpha_1$
for some $n\in\mathbb Z$. But then $\alpha_1-\gamma\not\in I$ since every other
element in $I$ has a factor of $\alpha_1$ in it. Thus, the second condition of
being an ideal is not met, so our initial assumption that $\gamma$ existed must
have been false.

This proves it, since every ideal in $\mathbb Z_8$ has a second-lowest element,
and we have shown that the ideal generated by that element cannot be improved
upon. Thus the 8 ideals listed above are all the ideals of $\mathbb Z_8$.

\textbf{7. Suppose $\phi$ is a ring homomorphism from $\mathbb Z\times\mathbb Z$
to $\mathbb Z\times\mathbb Z$. What are the possibilities for $\phi((1,0))$?
Make sure you list all possible values of $\phi((1,0))$, and for each value give
an example of a homomorphism that acheives it}

I am assuming that the binary operations on $\mathbb Z\times\mathbb Z$ are the
simple pairwise addition and multiplication.

Since $\phi$ is a homomorphism, we know that $\phi((1,0)(1,0))=\phi((1,0)) 
\phi((1,0))=\phi((1,0))$. Let $\phi((1,0))=(x_1,x_2)$; the previous statement
tells us that $x_1^2=x_1$ and $x_2^2=x_2$. Well, the only integers satisfying
these two expressions are $1$ and $0$, so $\phi((1,0))$ must be some pairwise
combination of those two elements. Since there's only four such combinations,
we can check all possibilities individually.

For $\phi((1,0))=(0,0)$ we use the homomorphism $\phi((a,b))=(b,b)$. We can
quickly see that this is a homomorphism; it maps $(1,1)$ to $(1,1)$, and
\begin{align*}
\phi((a,b)+(c,d))&=\phi((a+c,b+d))=(b+d,b+d)=(b,b)+(d,d)=\phi((a,b))+\phi((c,d))
\\ 
\phi((a,b)(c,d))&=\phi((ac,bd))=(bd,bd)=\phi((a,b))\phi((c,d))
\end{align*}

For $\phi((1,0))=(1,0)$, we simply take $\phi$ to be the identity homomorphism.

For $\phi((1,0))=(0,1)$, we use the homomorphism $\phi((a,b))=(b,a)$. Once
again, this maps $(1,1)$ to itself, and
\begin{align*}
\phi((a,b)+(c,d))=&\phi((a+c,b+d))=(b+d,a+c)=(b,a)+(d,c)=\phi((a,b))+\phi((c,d))
\\
\phi((a,b)(c,d))=&\phi((ac,bd))=(bd,ac)=(b,a)(d,c)=\phi((a,b))\phi((c,d))
\end{align*}

Finally, for $\phi((1,0))=(1,1)$, we use the homomorphism $\phi((a,b))=(a,a)$,
which is a homomorphism by an identical argument to the first one.

Thus, all four possibilities are valid homomorphisms, so we are done.

\textbf{8. Let $R$ be a commutative ring, and let $N\subseteq R$ be the set of
elements $x\in R$ such that $x^n=0$ for some positive integer $n$. Is $N$
necessarily an ideal of $R$? If yes, prove it. If no, find a specific ring $R$
such that $N$ is not an ideal.}

We claim that $N$ is, indeed, an ideal of $R$, and we verify the three
conditions of ideal-hood. Firstly, it is obvious that $0\in N$ since $0^1=0$ by
definition. Secondly, we wish to show that $N$ is closed under subtraction.
Consider two elements, $a,b\in N$. We know there exist positive integers $n,m$
such that $a^n=0$ and $b^m=0$, and we wish to show that there exists some $p$
so that $(a-b)^p=0$. Well, consider $p=n+m$. Using the binomial theorem, we
expand that to
\[
(a-b)^p=(a-b)^{n+m}=\sum_{j=0}^{n+m}{\binom{n+m}{j}(-1)^ja^{n+m-j}b^{j}}
\]

Now, we see that if $j<n$, then $n+m-j>m$ so that $a^{n+m-j}=0$,
thus nullifying that term. Conversely, if $j\geq n$, then $b^j=0$, also
nullifying the term. So, all the terms of the summation are $0$, thus $n+m$ is
a valid value for $p$, thus implying that $(a-b)\in N$.

Lastly, we show that for any positive integer $k$, $ka\in I$; but this is
obvious since $(ka)^n=k^na^n=0\implies ka\in N$. Thus, $N$ meets all the points
of the definition of an ideal.
\end{document}
