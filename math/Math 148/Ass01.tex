\documentclass[a4paper,10pt]{article}
\usepackage{amsmath}
\usepackage{fullpage}
\usepackage{amsfonts}
\newcommand{\R}{\ensuremath{\mathbb{R}}}
\newcommand{\C}{\ensuremath{\mathbb{C}}}
\newcommand{\Z}{\ensuremath{\mathbb{Z}}}
\newcommand{\M}{\ensuremath{\mathbb{M}}}
\renewcommand{\v}[1]{\ensuremath{\overrightarrow{#1}}}
%opening
\title{MATH 148: Assignment 1}
\author{Adrian Petrescu (\#20240298)}

\begin{document}
\maketitle

\textbf{1. Show that a constant function $f(x)=c$ is integrable over an interval $[a,b]$ and that $\int_a^bf=c(b-a)$. Give a proof based strictly on upper and lower sums.}

We know that $f$ is integrable over $[a,b]$ if and only if for any $\epsilon>0$, there is a partition $R$ such that we have $U(f,R)-L(f,R)<\epsilon$. In other words, we need
\begin{align*}
U(f,R)-L(f,R)=&\sum_{j=1}^n{\sup{f[x_{j-1},x_j](x_j-x_{j-1})}}-\sum_{j=1}^n{\inf{f[x_{j-1},x_j](x_j-x_{j-1})}}<\epsilon\\
%=&(x_j-x_{j-1})\left(\sum_{j=1}^n{\sup{f[x_{j-1},x_j]}}-\sum_{j=1}^n{\inf{f[x_{j-1},x_j]}}\right)
\end{align*}

However, by virtue of the fact that $f$ is constant, we know that $\inf{f}=\sup{f}=c$ over any interval. Thus we can write
\begin{align}
=&\sum_{j=1}^n{c(x_j-x_{j-1})}-\sum_{j=1}^n{c(x_j-x_{j-1})}\nonumber\\
=&c\left(\sum_{j=1}^n{(x_j-x_{j-1})}-\sum_{j=1}^n{(x_j-x_{j-1})}\right)=0<\epsilon
\end{align}

So we see that $f$ is indeed integrable. The interesting thing to note from equation $(1)$ for $f$ is that the upper and lower sums are equal regardless of which partition $R$ was used. Thus, in order to find the value of the integral $\int_a^b{f}$, we can use any partition we like, including the trivial one $R=\left\{a,b\right\}$. Hence, by our definition of the integral, its value is
\begin{align*}
\sup_P{L(f,P)}=&L(f,R)\\
=&\sum_{j=1}^n{\inf{f[x_{j-1},x_j](x_j-x_{j-1})}}
=c\sum_{j=1}^n{(x_j-x_{j-1})}
=c(b-a)
\end{align*}
And therefore $\boxed{\int_a^b{f}=c(b-a)}$.

\textbf{2. Show that the function $f(x)=x$ is integrable over $[0,1]$, and compute $\int_0^1{f}$.}

In order for $f$ to be integrable, for every $\epsilon>0$, there must exist some partition $R$ of $[0,1]$ such that $U(f,R)-L(f,R)<\epsilon$. In other words, we need
\begin{align*}
U(f,R)-L(f,R)=&\sum_{j=1}^n{\sup{f[x_{j-1},x_j](x_j-x_{j-1})}}-\sum_{j=1}^n{\inf{f[x_{j-1},x_j](x_j-x_{j-1})}}<\epsilon\\
\end{align*}

We know that the function $f$ is strictly increasing, and therefore $\inf{f[x_{j-1},x_j]}$ always occurs at $f(x_{j-1})$, while $\sup{f[x_{j-1},x_j]}$ always occurs at $f(x_j)$ regardless of the partition used. Thus we can write 
\begin{align*}
=&\sum_{j=1}^n{f(x_j)(x_j-x_{j-1})}-\sum_{j=1}^n{f(x_{j-1})(x_j-x_{j-1})}\\
=&\sum_{j=1}^n{\left(f(x_j)-f(x_{j-1})\right)(x_j-x_{j-1})}
\end{align*}

We now consider the partition $R$. For any given $\epsilon$, we will choose the partition \[R:0=x_0<x_1<...<x_n=1\] such that all $x_j-x_{j-1}<\epsilon$. In other words, the closer together we are required to bring the upper and lower sums, the more we must refine our partition.

With this new restriction, we continue from our previous equation
\begin{align*}
&\sum_{j=1}^n{\left(f(x_j)-f(x_{j-1})\right)(x_j-x_{j-1})}\\
\leq&\sum_{j=1}^n{f(x_j)-f(x_{j-1})\epsilon}\\
=&\epsilon\sum_{j=1}^n{f(x_j)-f(x_{j-1})}\\
=&\epsilon\left(f(x_n)-f(x_0)\right) 
=\epsilon(1-0)=\epsilon &&\text{ (by a telescoping sum)}
\end{align*}

And therefore, we see that $f$ is integrable over $[0,1]$.

To find the value of the integral, we assume without loss of generality that the partitions $P$ are uniformly distributed; that is, $(x_j-x_{j-1})=\frac1n$ is constant for any $j$. We can assume this is true, because for any partition $P$, there is some refinement $P'$ that is uniform (to obtain it, simply take a common refinement of $P$ and the uniform partition $P_s$ where $s=\min{(x_j-x_{j-1})}$). We must evaluate
\begin{align*}
\inf_P{U(f,P)}=&\inf_P{\sum_{j=1}^n{\sup{f[x_{j-1},x_j](x_j-x_{j-1})}}}\\
=&\inf_P{\sum_{j=1}^n{f(x_j)(x_j-x_{j-1})}}\\
=&\inf_P{\sum_{j=1}^n{\frac{j}{n}\frac1n}}=\inf_P{\sum_{j=1}^n{\frac{j}{n^2}}}\\
=&\inf_P{\frac12\cdot\frac{n+1}{n}}
\end{align*}

We see that as $n\to\infty$, $\frac{n+1}{n}\to1$, so the infinimum is $\frac12$ and thus \[\boxed{\int_0^1{f}=\frac12}\]

\textbf{4. Prove that the function $f(x)=\frac{\sin{x}}{1+x^2}$ is uniformly continuous over $\R$.}

We know that $f$ is uniformly continuous on $\R$ if for every $\epsilon>0$ there is some $\delta>0$ such that, for all $x,a\in\R$,
\[|f(x)-f(a)|<\epsilon \text{ whenever } |x-a|<\delta\]


\textbf{5. Prove that the function $f(x)=\frac1x$ is not uniformly continuous over the open interval $(0,1)$.}

We know that $f$ is continuous on $(0,1)$, namely that for any $x,p\in(0,1),|x-p|<\delta\implies\left|\frac1x-\frac1p\right|<\epsilon$. but we would like to show that the choice of $\delta$ is a function of $p$.
\end{document}
