\documentclass[11pt]{article}
\usepackage{geometry}             
\geometry{letterpaper}                   % ... or a4paper or a5paper or ... 
%\geometry{landscape}                % Activate for for rotated page geometry
\usepackage{fullpage}
\usepackage[parfill]{parskip}    % Activate to begin paragraphs with an empty line rather than an indent
\usepackage{graphicx}
\usepackage{amssymb}
\usepackage{epstopdf}
\usepackage{amsmath}
\DeclareGraphicsRule{.tif}{png}{.png}{`convert #1 `dirname #1`/`basename #1 .tif`.png}

\title{Math 148 � Assignment 7}
\author{Adrian Petrescu (\#20240298)}
%\date{}                                           % Activate to display a given date or no date

\begin{document}
\maketitle

\textbf{1. (a) Show that
\[
-x^{2n}\leq\frac{1}{1+x^2}-\left(1-x^2+x^4-x^6+\cdots+(-1)^{n-1}(x^2)^{n-1}\right)\leq x^{2n}
\]
for each positive integer $n$.
}

We see that the terms in the brackets form a geometric series with first term of $a=1$ and a common ratio of $-x^2$. We have our formula for the sum of a geometric series:
\[
S_n=\sum_{i=0}^n{ar^k}=\frac{a(1-r^{n+1})}{1-r}
\]

So by substitution we have
\begin{align*}
S_n=\frac{1+x^{2n+2}}{1+x^2}=\left(-x^2+x^4-x^6+\cdots+(-1)^{n-1}(x^2)^{n-1}\right)
\end{align*}
So we put this into the inequality, and simplify:
\begin{align*}
-x^{2n}\leq\frac{1}{1+x^2}-\left(\frac{1+x^{2n+2}}{1+x^2}\right)\leq x^{2n}\\
-x^{2n}\leq-\left(\frac{x^{2}\cdot x^{2n}}{1+x^2}\right)\leq x^{2n}\\
x^{2n}\geq x^{2n}\left(\frac{x^2}{1+x^2}\right) \geq-x^{2n} 
\end{align*}
We see that the bracketed quantity is always less than $1$, so the left inequality holds. Furthermore, the quantity in the center is always positive, so the right inequality holds.

\textbf{1. (b) Do an integration over $[0,1]$ of the above functions and compute the sum of the series
\[ 
1-\frac13+\frac15-\frac17+\frac19-\frac1{11}+\frac1{13}-\cdots
\]
}
It is immediately obvious that $\displaystyle\int_0^1\pm x^{2n}dx=\pm\left[\frac{x^{2n+1}}{2n+1}\right]_0^1=\pm\frac{1}{2n+1}$. We can also see that
\begin{align*}
&\int{\left[\frac{1}{1+x^2}-1+x^2-x^4+x^6+\cdots(-1)^n(x^2)^n \right]}\\
=&\int\frac{1}{1+x^2}-\int1+\int x^2-\int x^4+\int x^6+\cdots+(-1)^n\int x^{2n}\\
=&\arctan{x}-x+\frac{x^3}{3}-\frac{x^5}{5}+\frac{x^7}{7}+\cdots+(-1)^n\frac{x^{2n+1}}{2n+1}
\end{align*}
At $x=0$, all of the terms above are equal to $0$, so if we take the integral over the interval $[0,1]$, all we must do is substitute $x=1$ into the above equation. Thus we have the inequality
\[
-\frac{1}{2n+1}\leq\arctan{1}-1+\frac13-\frac15+\frac17+\cdots+\frac{(-1)^n}{2n+1}\leq\frac{1}{2n+1}
\]

If we take the limit of all three sides as $n\to\infty$, we have
\[
0\leq\frac\pi4-1+\frac13-\frac15+\frac17+\cdots\leq0
\]
which implies our final answer
\[
1+\frac13-\frac15+\frac17+\cdots=\frac\pi4
\]

\textbf{2. Find the radius of convergence of the following series:}

\textbf{(a) $\displaystyle\sum_{n=2}^\infty{\frac{1}{(\ln{n})^2}x^n}$}

We use the root test:
\begin{align*}
\lim_{n\to\infty}{\sqrt[n]{\frac{1}{(\ln{n})^2}x^n}}=x\lim_{n\to\infty}{\ln{(n)}^{-\frac{2}{n}}}=x\lim_{n\to\infty}{\left(e^{-2\frac{\ln\ln{n}}{n}}\right)}=x\cdot e^{-2\lim_{n\to\infty}{\frac{\ln\ln{n}}{n}}}
\end{align*}
Both the numerator and denominator of the limit go to $\infty$, so we can apply L'Hopital's rule and take the derivative:
\begin{align*}
x\cdot e^{-2\lim_{n\to\infty}{\frac{1}{n\ln{n}}}}=x\cdot e^0=x
\end{align*}
So we see that the limit converges if $x<1$; in the indeterminate case $x=1$, the series is simply
\[
\sum_{n=2}^\infty{\frac{1}{\ln{n}^2}}\sim\sum_{k=1}^\infty{\frac{2^k}{\ln{(2^k)}^2}}=\sum_{k=1}^\infty{\frac{2^k}{k^2\ln2^2}}=\frac{1}{\ln{2}^2}\sum_{k=1}^\infty{\frac{2^k}{k^2}}
\]
which clearly diverges by the condensation test. Thus the radius of convergence is $R=1$.

\textbf{(b) $\displaystyle\sum_{n=2}^\infty{\frac{1}{(\ln{n})^n}{x^n}}$}

We apply the root test again:
\begin{align*}
\lim_{n\to\infty}{\sqrt[n]{\frac{1}{(\ln{n})^n}x^n}}=\lim_{n\to\infty}{\frac{x}{\ln{n}}}
\end{align*}

This converges regardless of $x$; thus the radius of convergence for this series is simply $\mathbb{R}$.

\textbf{(c) $\displaystyle\sum_{n=3}^\infty{\frac{1}{(\ln{n})^{\ln{n}}}x^n}$}

We once again apply the root test, and proceed almost exactly as in part (a):
\begin{align*}
\lim_{n\to\infty}{\sqrt[n]{\frac{1}{(\ln{n})^{\ln{n}}}x^n}}=x\lim_{n\to\infty}{\ln{(n)}^{-\frac{\ln{n}}{n}}}=x\lim_{n\to\infty}{\left(e^{-\ln{n}\frac{\ln\ln{n}}{n}}\right)}=x\cdot e^{-\lim_{n\to\infty}{\frac{\ln{n}\ln\ln{n}}{n}}}
\end{align*}

Once again we see that both the numerator and the denominator go to $\infty$, so we can apply L'Hopital's rule (twice) to the limit to get
\begin{align*}
x\cdot e^{-\lim_{n\to\infty}{\frac{\ln{\ln{n}}+1}{n}}}=x\cdot e^{-\lim_{n\to\infty}{\frac{1}{n\ln{n}}}}=x\cdot e^0=x
\end{align*}

Thus once again, the limit is less than $1$ when $x<1$. For the indeterminate case, the series is simply
\[
\sum_{n=3}^\infty{\frac{1}{(\ln{n})^{\ln{n}}}}
\]
which by the condensation test is equivalent to
\begin{align*}
\sum_{k=2}^\infty{\frac{2^k}{(\ln{2^k})^{\ln{2^k}}}}=\sum_{k=2}^\infty{\left(\frac{2}{(\ln{2^k})^{\ln2}}\right)^k}
\end{align*}
which, by the root test, converges if the limit
\begin{align*}
\lim_{k\to\infty}{\sqrt[k]{\left(\frac{2}{(\ln{2^k})^{\ln2}}\right)^k}}=\lim_{k\to\infty}{\frac{2}{(k\ln2)^{\ln2}}}
\end{align*}
is less than 1, which it clearly is (it goes to $0$). Thus when $x=1$, the series converges, so the radius of convergence is $R=1$.

\textbf{3. Suppose $x_n$ is a bounded sequence and $c$ is a fixed positive constant. Prove that $\limsup{cx_n}=c\limsup{x_n}$.}

Consider $p=\limsup{x_n}$; that is equivalent to the statement that, for every $\epsilon>0, x_n<p+\epsilon$ eventually. Then, the the statement that $c\limsup{x_n}=cp=\limsup{cx_n}$ is the statement that, for every $\epsilon>0, cx_n<cp+\epsilon$. Dividing both sides by $p$ (which we can do, even in the inequality, because $c$ is a positive real number by assumption), we get $x_n<p+\frac\epsilon c$, which we know to be true by assumption (we can ignore the difference between $\epsilon$ and $\frac\epsilon c$ since they are both just arbitrarily small positive numbers); thus it implies the required statement.

\textbf{4. Suppose $\sum_{n=0}^\infty{a_nx^n}$ has a radius of convergence $R$, where $0<R<\infty$, and let $k$ be a fixed positive integer. What is the radius of convergence of $\sum_{n=0}^\infty{a_nx^{km}}$? What is the radius of convergence of $\sum_{n=0}^\infty{a_nx^{n^2}}$?}

If the radius of convergence is $R$, some positive value, that means, by the root test,
\begin{align*}
x\cdot&\lim_{n\to\infty}{\sqrt[n]{a_n}}<1\quad\text{whenever }x<R\\
&\lim_{n\to\infty}{\sqrt[n]{a_n}}<\frac1x\quad\text{whenever }x<R
\end{align*}

Similarly, by the comparison test,
\begin{align*}
\lim_{n\to\infty}{\left|\frac{a_{n+1}}{a_n}\right|}<\frac1x\quad\text{whenever }x<R
\end{align*}
Now, for $\sum_{n=0}^\infty{a_nx^{kn}}$, the ratio test tells us that the series converges if
\[
x^k\cdot\lim_{n\to\infty}{\left|\frac{a_{n+1}}{a_n}\right|}<1\implies\lim_{n\to\infty}{\left|\frac{a_{n+1}}{a_n}\right|}<\left(\frac1x\right)^k
\implies\sqrt[k]{\lim_{n\to\infty}{\left|\frac{a_{n+1}}{a_n}\right|}}<\frac1x
\]
We already know that $\frac1x$ is greater than the limit when $x<R$; in order for it to be greater than the $n$th root of the limit, we must have $x<R'=\sqrt[n]R$.

For the second series, we apply the root test:
\[
\lim_{n\to\infty}{x^n\sqrt[n]{a_n}}<1
\]

\textbf{5. Think of a power series whose radius of convergence is $1$ and converges conditionally at both $\pm1$.}

The simplest example is
\[
\sum_{n=1}^\infty{\frac{(-1)^n}{n}x^n}
\]

By applying the ratio test, we see
\begin{align*}
\lim_{n\to\infty}{\left|\frac{a_{n+1}}{a_n}\right|}=\lim_{n\to
infty}{\left|\frac{n+1}{n}x\right|}=|x|\lim_{n\to\infty}{\frac{n+1}{n}}<1\implies|x|<1
\end{align*}

However, at the equality $x=\pm1$, we simply have the series
\[
\pm1\mp\frac1x\pm\frac12\mp\frac13\pm\frac14\mp\frac15\pm\cdots
\]

Which is the classic example of a series which converges, but not absolutely (since $\sum{\frac1n}$ diverges).

\textbf{6. Find power series expansions for each of the following functions.}

\textbf{(a) $\displaystyle\frac{1}{2x^2-3x+1}$.}

Using partial fractions, we decompose
\begin{align*}
&\frac{1}{2x^2-3x+1}=\frac{1}{(2x-1)(x-1)}=\frac{A}{2x-1}+\frac{B}{x-1}\\
\implies&A(x-1)+B(2x-1)=1\implies(A+2B)x-A-B=1\\\implies&A=-2, B=1
\end{align*}
So we now have
\[
\frac{-2}{2x-1}+\frac1{x-1}
\]

Each of those can be expressed as a summation of an infinite series of the form
\[
S=\sum_{k=0}^\infty{ar^k}=\frac{a}{1-r}
\]

The first one clearly has $a=-2$ and $1-r=2x-1\implies r=2-2x$. The second one has $a=1$ and $1-r=x-1\implies r=2-x$. So we can write
\[
\boxed{\frac{1}{2x^2-3x+1}=\sum_{k=0}^\infty{(-2)(2-2x)^k}+\sum_{k=0}^\infty{(2-x)^k}=\sum_{k=0}^\infty{(-2)(2-2x)^k+(2-x)^k}}
\]

\textbf{(b) $\displaystyle\frac{1}{x^3-2x^2+x-2}$}

The polynomial in the denominator factors into $(x-2)(x^2+1)$. Thus by partial fractions,
\begin{align*}
&\frac{A}{x-2}+\frac{Bx+C}{x^2+1}=\frac{A(x^2+1)+(Bx+C)(x-2)}{(x-2)(x^2+1)}\\
=&\frac{Ax^2+A+Bx^2-2Bx+Cx-2C}{(x-2)(x^2+1)}\\
=&\frac{(A+B)x^2+(C-2B)x+(A-2C)}{(x-2)(x^2+1)}=\frac{1}{(x-2)(x^2+1)}
\end{align*}

Solving this system of equations, we obtain
\[
A=\frac15,B=-\frac15,C=-\frac25
\]
Thus
\[
\frac{1}{x^3-2x^2+x-2}=\frac{1}{5x-10}+\frac{-x-2}{5x^2+5}
\]

\textbf{7. If $\displaystyle\sum_{n=0}^\infty{a_nx^n}$ has radius $R$, prove that the derived series $\displaystyle\sum_{n=1}^\infty{na_nx^{n-1}}$ has the same radius $R$.}

It is sufficient to prove that $\sum_{n=1}^\infty{na_nx^{n-1}}$ converges$\implies\sum_{n=0}^\infty{a_nx^n}$ converges, and that, for any $y$ such that $|y|<|x|$, then we have $\sum_{n=0}^\infty{a_nx^n}$ converges$\implies\sum{na_ny^{n-1}}$ converges.\footnote{Thanks to Richard Liang for this insight and some guidance.} This proves it, because if they did not have the same radius of convergence, then there would be some $R_2<|x|<R_1$ such that the first series converges, but the derivative series does not; this contradicts the second condition. Conversely, there may be some $R_2>|x|>R_1$ so that the derivative series converges, but the original does not; however, that contradicts the first condition. Since there is no $|x|$ between $R_1$ and $R_2$, we must conclude that $R_1=R_2$.

To prove the first one, we recall from a previous assignment that if $\sum{x_n}$ converges, then so does $\sum{nx_n}$. So we set up a comparison
\[
|a_nx^n|\leq|na_nx^{n-1}|\cdot|x|
\]

Thus if the derivative converges, then multiplying it by $x$ will not change that; this whole thing is bigger than the original series (by a factor of $|n|$ exactly), and so by the comparison test, they both converge. So the first part is proven.

To prove the second part, take $y$ such that $|y|<|x|$. We know that
\[
\sum_{n=0}^\infty{n\left(\frac{|y|}{|x|}\right)^n}
\] 
converges with $R=1$ (by a very simple application of the ratio test), and therefore the sequence is also bounded. Let $v$ be some bound for the series; then $n\frac{|y|}{|x|}^n<v\implies n|y|^n<v|x|^n$.

Now, we want to show that $\sum_{n=1}^\infty{na_ny^{n-1}}$ converges; well, if we substitute $|yx|^n$ with $|vx|^n$, we get a bigger series
\[
\sum_{n=0}^\infty{|a_nx^n|\cdot\frac{|v|}{|y|}}=\frac{|v|}{|y|}\cdot\sum_{n=0}^\infty{|a_nx^n|}
\]

We know by assumption that this series converges. So we have constructed a larger, convergent series; by the comparison test, the derivative series converges also; so both conditions have been proven.

\textbf{8. (a) Just to warm up, calculate and simplify $\displaystyle\binom{-2}{6},\binom{1/2}{5},\binom{-1/3}{4}$.}

By definition,
\[
\binom{\alpha}{r}=\frac{\alpha(\alpha-1)(\alpha-3)\cdots(\alpha-r+1)}{r!}
\]
So by simple substitution, we have
\[
\boxed{\binom{-2}{6}=\frac{(-2)(-3)(-4)(-5)(-6)(-7)}{6!}=(-1)^6\frac{7!}{6!}=7}
\]
\[
\boxed{\binom{1/2}{5}=\frac{(1/2)(-1/2)(-3/2)(-5/2)(-7/2)}{5!}=(-1)^4\frac{7\cdot5\cdot3}{2^5\cdot5!}=\frac7{256}}
\]
\[
\boxed{\binom{-1/3}{4}=\frac{(-1/3)(-4/3)(-7/3)(-10/3)(-13/3)}{4!}=(-1)^5\cdot\frac{4\cdot7\cdot10\cdot13}{3^5\cdot4!}=-\frac{455}{729}}
\]

\textbf{8. (b) Verify by grinding out the definition of binomial coefficients that
\[
(r+1)\binom{\alpha}{r+1}+r\binom{\alpha}{r}=\alpha\binom{\alpha}{r}
\] for all $r=0,1,2,\cdots$}

We will just bash...
\begin{align*}
&(r+1)\frac{\alpha(\alpha-1)(\alpha-2)\cdots(\alpha-r)}{(r+1)!}+r\frac{\alpha(\alpha-1)(\alpha-2)\cdots(\alpha-r+1)}{r!}\\
=&\frac{\alpha(\alpha-1)(\alpha-2)\cdots(\alpha-r)}{r!}+\frac{\alpha(\alpha-1)(\alpha-2)\cdots(\alpha-r+1)}{(r-1)!}\\
%=&\frac{\alpha(\alpha-1)(\alpha-2)\cdots(\alpha-r+1)}{(r-1)!}\cdot\left(1+\frac{(\alpha-r)}{r}\right)
=&\frac{\alpha(\alpha-1)(\alpha-2)\cdots(\alpha-r)}{r!}+\frac{r\alpha(\alpha-1)(\alpha-2)\cdots(\alpha-r+1)}{r!}\\
=&\frac{\alpha(\alpha-1)(\alpha-2)\cdots(\alpha-r)+r\alpha(\alpha-1)(\alpha-2)\cdots(\alpha-r+1)}{r!}\\
=&\left(\frac{\alpha(\alpha-1)(\alpha-3)\cdots(\alpha-r+1)}{r!}\right)\left(\alpha-r+r\right)\\
=&\alpha\cdot\left(\frac{\alpha(\alpha-1)(\alpha-3)\cdots(\alpha-r+1)}{r!}\right)\\
=&\alpha\binom{\alpha}{r}
\end{align*}

\textbf{8. (c) For each $\alpha$ use the ratio test to show that the radius of convergence of the series
\[
1+\alpha x+\binom{\alpha}{2}x^2+\binom{\alpha}{3}x^3+\cdots+\binom{\alpha}{r}+x^r+\cdots
\] is $1$. Where did you need the fact that $\alpha\not=0,1,2,\cdots$?}

We apply the ratio test
\begin{align*}
&\lim_{n\to\infty}{\left|\frac{\binom{\alpha}{n+1}}{\binom{\alpha}{n}}\right||x|}\\
=&|x|\lim_{n\to\infty}{\left|\frac{\frac{\alpha(\alpha-1)(\alpha-2)\cdots(\alpha-n)}{(n+1)!}}{\frac{\alpha(\alpha-1)(\alpha-2)\cdots(\alpha-n+1)}{n!}}\right|}\\
=&|x|\lim_{n\to\infty}{\left|\frac{\alpha-n}{n+1}\right|}\\
=&|x|\left(\lim_{n\to\infty}{\left|\frac{\alpha}{n+1}\right|}-\lim_{n\to\infty}{\left|\frac{n}{n+1}\right|}\right)\\
=&|x||(0-1)|=|x|<1
\end{align*}

Thus the radius of convergence is $R=1$; we never used the fact that $\alpha$ is not an integer; this implies that it works for regular integral binomial coefficients also.

\textbf{(d) If $\displaystyle f(x)=\sum_{r=0}^\infty{\binom{\alpha}{r}x^r}$ where $x\in(-1,1)$, show that
\[
(1+x)f'(x)=\alpha f(x)\quad\forall x\in(-1,1)
\]}

By the differentiation formula, we have
\[
f'(x)=\sum_{r=1}^\infty{r\binom{\alpha}{r}x^{r-1}}
\]

We do some expansion:
\begin{align*}
f'(x)+xf'(x)=&\alpha f(x)\implies f'(x)+\sum_{r=1}^\infty{r\binom{\alpha}{r}x^r}=\alpha f(x)
\end{align*}

Then, using the identity derived in part (b), we obtain
\begin{align*}
f'(x)+\sum_{r=0}^\infty{\alpha\binom{\alpha}{r}x^r}-\sum_{r=0}^\infty{(r+1)\binom{\alpha}{r+1}x^r}=\alpha f(x)\\
\implies f'(x)=\sum_{r=0}^\infty{(r+1)\binom{\alpha}{r+1}x^r}\\
\implies \sum_{r=1}^\infty{r\binom{\alpha}{r}x^{r-1}}=\sum_{r=0}^\infty{(r+1)\binom{\alpha}{r+1}x^r}
\end{align*}

This may at first seem to be false, but this is due to the discrepancy in the starting index of the summations; if we take $s=r+1$, we see that
\[
 \sum_{r=1}^\infty{r\binom{\alpha}{r}x^{r-1}}=\sum_{s=1}^\infty{s\binom{\alpha}{s}x^{s-1}}=\sum_{r=0}^\infty{(r+1)\binom{\alpha}{r+1}x^r}
\]

and these are, in fact, always equivalent. Thus the left and right sides are always equivalent, and we have proven the identity.

\textbf{(e) Prove that $\displaystyle\left(\frac{f(x)}{(1+x)^\alpha}\right)'=0$, and use this result to deduce that
\[
f(x)=(1+x)^\alpha\quad\text{for }-1<x<1
\]}

We take the derivative:
\begin{align*}
&\left(\frac{f(x)}{(1+x)^\alpha}\right)'=\frac{f'(x)(1+x)^\alpha-f(x)\left(\frac{\alpha(1+x)^\alpha}{x+1}\right)}{(1+x)^{2\alpha}}\\
=&\frac{\alpha f(x)(1+x)^{\alpha-1}-\alpha f(x)(x+1)^{\alpha-1}}{(1+x)^{2\alpha}}\\
=&\frac{\alpha f(x)((1+x)^{\alpha-1}-(1+x)^{\alpha-1})}{(1+x)^{2\alpha}}\\
=0
\end{align*}

This tells us that the function we were taking the derivative of is a constant function; so if we evaluate it at any point, we will know its value over the entire domain. Thus, we look at $x=0$, since it is the easiest to evaluate, and we get
\[
\frac{f(0)}{(1+0)^\alpha}=\frac{1}{1}=1
\]

Thus the function is equal to $1$ over its whole domain. So, rearranging, we get
\[
\frac{f(x)}{(1+x)^\alpha}=1\implies \boxed{f(x)=(1+x)^\alpha}
\]

\textbf{(f) Find the binomial expansion of $\frac{1}{(1+x)^2}$. Explain why the answer is the same as the series that represents
\[
(-1+x-x^2+x^3-\cdots+(-1)^{n+1}x^n+\cdots)'.
\]}

We know
\[
f(x)=(1+x)^\alpha=\sum_{r=0}^\infty{\binom{\alpha}{r}x^r}
\]

So for the specific case $\alpha=-2$, we have the binomial expansion
\[
\boxed{\frac{1}{1+x^2}=\sum_{r=0}^\infty{\binom{-2}{r}x^r}}.
\]

We look at the binomial coefficient $\binom{-2}{r}$:
\[
\binom{-2}{r}=\frac{(-2)(-3)(-4)\cdots(-r)(-r-1)}{r!}=(-1)^{r}(r+1)
\]

We also look at:
\[
(-1+x-x^2+x^3-\cdots+(-1)^{n+1}x^n+\cdots)'=1-2x+3x^2-4x^3+\cdots+(-1)^{n}(n+1)x^n=\sum_{r=0}^\infty{(-1)^r(r+1)x^r}
\]

We see that the binomial coefficient $\binom{-2}{r}$ matches up perfectly with the power series expansion shown above.

\end{document}