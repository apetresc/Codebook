\documentclass[11pt]{article}
\usepackage{geometry}             
\geometry{letterpaper}                   % ... or a4paper or a5paper or ... 
%\geometry{landscape}                % Activate for for rotated page geometry
\usepackage{fullpage}
\usepackage[parfill]{parskip}    % Activate to begin paragraphs with an empty line rather than an indent
\usepackage{graphicx}
\usepackage{amssymb}
\usepackage{epstopdf}
\usepackage{amsmath}
\DeclareGraphicsRule{.tif}{png}{.png}{`convert #1 `dirname #1`/`basename #1 .tif`.png}

\title{Math 148 � Assignment 6}
\author{Adrian Petrescu (\#20240298)}
%\date{}                                           % Activate to display a given date or no date

\begin{document}
\maketitle

\textbf{1. Use a limit comparison with $\sum{\frac{1}{n^2}}$ to show that $\sum{(1-\cos{\frac1n})}$ converges.}

The limit comparison test says that if $\{a_n\},\{b_n\}$ are two sequences that converge to $0$, and the limit $\lim{\frac{a_n}{b_n}}$ exists and is not equal to $0$, then $\sum{a_n}$ converges if and only if $\sum{b_n}$ converges. Well, we know $\frac1{n^2}$ converges to $0$, and as $n\to\infty, \frac1n\to0\implies\cos\frac1n\to1\implies1-\cos\frac1n\to0$ also. Thus, we should evaluate the limit
\[
\lim_{n\to\infty}{\left[n^2\left(1-\cos\frac1n\right)\right]}
\]
Consider $x=\frac1n$. As $n\to\infty,x\to0$. Thus we can rewrite this limit as
\[
\lim_{x\to0}{\frac{1-\cos{x}}{x^2}}
\]

We see that as $x\to0$, both the numerator and denominator approach $0$, so L'Hopital's rule applies. Taking the derivative, we obtain
\[
\lim_{x\to0}{\frac{\sin{x}}{2x}}
\]

Once again, as $x\to0$, both the numerator and denominator approach 0, so we take the derivative again:
\[
\lim_{x\to0}{\frac{\cos{x}}{2}}=\frac12
\]

Thus the limit exists and is not equal to $0$, so $\sum{1-\cos{\frac1n}}$ converges if $\sum{\frac1{n^2}}$ converges, which we know it does.

\textbf{2. Show that $\displaystyle\sum_{k=3}^\infty{\frac1{k\ln{k}}}$ diverges. If $\displaystyle\sum_{k=3}^\infty{\frac1{k\ln{k}}}\geq10$, prove that $n\geq e^{e^{10}}-1$.}

We will first show that the series diverges by comparing it to the integral
\[
\int_3^\infty{\frac{dx}{x\ln{x}}}
\]

We fix the differential by looking at $u=\ln{x}$. Then $du=\frac1xdx$, so by substitution, the integral is equivalent to
\[
\int_3^\infty{\frac1udu}=\left[\ln{u}\right]_3^\infty=\left[\ln\ln{x}\right]_3^\infty
\]

We know $\ln$ is not bounded, so as $x\to\infty,\ln{x}\to\infty\implies\ln\ln{x}\to\infty$, and so the integral does not exist, and thus the series diverges.

Now, we know that the series corresponds to the Riemann sum where the partition is all the integer values in $[3,\infty)$, which means that all the rectangles are of width $1$. Since $\ln{\ln{x}}$ is clearly an increasing function everywhere on $[3,\infty)$, the series necessarily underestimates the integral. Thus, when the series is above $10$, we have the useful inequality
\[
10\leq\sum_{k=3}^n{\frac1{k\ln{k}}}<\int_3^n{\frac{dx}{x\ln{x}}}=\left[\ln\ln{x}\right]_3^n
\]

We are trying to show that $n\geq e^{e^{10}}-1$. We also know that $e^x$ is an increasing function, so $x>y\iff e^x>e^y$. Keeping that in mind, we do some algebra:
\begin{align*}
&\ln\ln{n}-\ln\ln3>10\implies\ln\ln{n}>10+\ln\ln3\\
\iff&e^{\ln\ln{n}}>e^{10+\ln{\ln{3}}}\implies\ln{n}>e^{10}e^{\ln\ln3}=e^{10}\ln3\\
\iff&e^{\ln{n}}>e^{e^{10}\ln3}\implies n>\left(e^{\ln3}\right)^{e^{10}}=3^{e^{10}}
\end{align*}

Since $e<3$, it is obvious that $3^{e^{10}}>e^{e^{10}}$, and so we have shown that
\[
\boxed{n>e^{e^{10}}-1}
\] 

\textbf{We know that
\[
0<e-\left(1+1+\frac1{2!}+\frac1{3!}+\cdots+\frac{1}{n!}\right)<\frac{1}{n!n}.
\] Prove that $e$ is irrational.
}

For the sake of simplicity, we use
\[
\sum_{k=0}^n{\frac{1}{k!}}=\left(1+1+\frac1{2!}+\frac1{3!}+\cdots+\frac{1}{n!}\right)
\]

Then we can rewrite the given inequality as
\[
\sum_{k=0}^n{\frac{1}{k!}}<e<\frac{1}{n!n}+\sum_{k=0}^n{\frac{1}{k!}}
\]

Let us assume that $e$ is rational and can be written as $\frac{a}{b}$ where $a,b\in\mathbb{Z^+}$, and seek a contradiction. Well, as $n\to\infty$ on all three sides, we have $\frac{1}{n!n}\to0$, and thus by the squeeze principle,
\[
\frac{a}{b}=\sum_{k=0}^\infty{\frac{1}{k!}}
\]

We multiply both sides by $b!$ to obtain
\[
a(b-1)!=\sum_{k=0}^\infty{\frac{b!}{k!}}
\]

Now, in general, $\frac{m!}{n!}$ is an integer if $m\geq n$. Since $b$ is a positive integer, and $k$ iterates through all positive integers, we can write the sum as
\begin{align*}
a(b-1)!&=\sum_{k=0}^{b}{\frac{b!}{k!}}+\sum_{k=b+1}^\infty{\frac{b!}{k!}}\\
a(b-1)!-\sum_{k=0}^{b}{\frac{b!}{k!}}&=\sum_{k=b+1}^\infty{\frac{b!}{k!}}=\frac{1}{b+1}+\frac{1}{(b+1)(b+2)}+\cdots
\end{align*}

It is clear that the left side is an integer, and that the right side is positive. Thus the right side is a positive integer. However, we also know
\[
\frac{1}{b+1}+\frac{1}{(b+1)(b+2)}+\cdots<\frac{1}{b+1}+\frac{1}{(b+1)^2}+\cdots=\sum_{j=1}^\infty{\frac{1}{(b+1)^j}}=\frac{1}{b}
\]

Since $b$ is an integer, the right-hand side is at most $1$, so the left hand side is less than $1$... but that contradicts the fact that it is a positive integer, since there are no positive integers less than 1. Thus, our assumption that $e=\frac{a}{b}$ must have been faulty, and so $e$ is irrational.

\textbf{4. (a) Show that the series $\displaystyle\sum_{n=1}^\infty{\frac{n!}{n^n}x^n}$ converges absolutely when $|x|<e$ and diverges when $|x|>e$.}

We know $n$ is a positive integer, so $n!$ and $n^n$ are also positive. Thus the series converges absolutely if
\[
\sum_{n=1}^\infty{\frac{n!}{n^n}|x|^n}
\]
converges. By the limit test for convergence, we wish to examine
\[
\lim_{n\to\infty}{\sqrt[n]{\frac{n!}{n^n}|x|^n}}=\lim_{n\to\infty}{\left[\sqrt[n]{\frac{|x|^n}{n^n}}\sqrt[n]{n!}\right]}=\lim_{n\to\infty}{\left[|x|\cdot\frac{\sqrt[n]{n!}}{n}\right]}=|x|\lim_{n\to\infty}{\frac{\sqrt[n]{n!}}{n}}
\]

Assuming we know the famous identity
\[
e=\lim_{x\to\infty}{\frac{x}{\sqrt[x]{x!}}}\implies\frac1e=\lim_{x\to\infty}{\frac{\sqrt[x]{x!}}{x}}
\]
So our limit evaluates to \[\lim_{n\to\infty}{\sqrt[n]{a_n}}=\frac{|x|}{e}\]

By the root test, if $|x|<e$, the limit is less than $1$, so the series converges. If $|x|>e$, the limit is greater than $1$, so the series diverges. As a note, when $|x|=e$, the limit is exactly 1, so the root test is indeterminate.

\textbf{4. (b) Use a simple diagram to explain the inequality
\[
\int_0^n{\ln{(t)}dt}<\ln2+\ln3+\cdots+\ln{n}\quad\text{for }n=2,3,4,_{\cdots}
\]
}
Since $\int_0^1{\ln{(t)}dt}$ is negative, it suffices to show that $\int_1^n{\ln{(t)}dt}<\sum_{k=1}^n{\ln{k}}$, since the integral from $0$ is even smaller than the integral from $1$.
\begin{figure}[htp]
\centering
\includegraphics[totalheight=0.5\textheight]{ln_inequality.eps}
\caption[$\ln$ inequality]
{$\ln$ inequality}\label{fig:ln_inequality}
\end{figure}

\textbf{4. (c) Show that $n^ne^{1-n}<n!$ for $n=2,3,...$.}

We start with a bit of rearrangement of the inequality:

\[
n!>n^n\frac{e}{e^n}\implies\frac{n!}{n^n}>\frac{e}{e^n}\implies\frac{n!}{n^n}e^n>e
\]
For $n=2$, we have $\frac{e^2}{2}>e$ which is clearly true. We will proceed by mathematical induction on $n$; assume it is true for all $k\leq n$, prove it true for $n+1$. Thus we are trying to prove that
\[
\frac{(n+1)!}{(n+1)^{n+1}}e^{n+1}=\frac{n!}{(n+1)^n}e^{n+1}>e\implies\frac{n!e^n}{(n+1)^n}>1
\]

We know from the inductive hypothesis that $n!e^n>en^n$, so it suffices to show that
\[
\frac{en^n}{(n+1)^n}>1\iff e>\left(\frac{n+1}{n}\right)^n
\]

Well, we know that the actual definition of $e$ is
\[
e=\lim_{n\to\infty}{\left(1+\frac1n\right)^n}
\]

So if we can show that the derivative of that function is always positive, we know that it is always growing and approaching the value $e$ (but it is therefore always below it). So, taking the derivative, we see
\[
\left[\left(1+\frac1n\right)^n\right]'=\left[e^{n\ln{\left(1+\frac1n\right)}}\right]'=\left(n\ln{\left(1+\frac1n\right)}\right)'\cdot e^{n\ln{\left(1+\frac1n\right)}}
\]
The second factor is always positive, since $e^x>0$ always, so we only need to worry about the first. Continuing the differentiation we get
\[
\left(n\ln{\left(1+\frac1n\right)}\right)'=\ln{\left(1+\frac1n\right)}+n\left(\frac{1}{1+\frac{1}{n}}\right)
\]

Clearly for $n>0$ this is positive. Thus the derivative is always positive, so the function above is always growing, and since it approaches $e$, it must always be below $e$ (if it were above $e$ and growing, it wouldn't approach $e$). Thus we have proven the induction for $n+1$, and thus it is always true.

\textbf{4. (d) Decide on the convergence of $\displaystyle\sum_{n=1}^\infty{\frac{n!e^n}{n^n}}$ and $\displaystyle\sum_{n=1}^\infty{\frac{n!(-e)^n}{n^n}}$.}

We have just proven on the previous question that \[ \frac{e^nn!}{n^n}>e\quad\forall n\geq2\] and thus the terms of the sequence do not even approach 0. Thus the first series diverges very, very quickly.

For the second series, we employ the alternating series test. The series converges if $\left\{\frac{n!e^n}{n^n}\right\}$ approaches $0$, which we have just shown it does not, since all terms are greater than $e$. Thus the second series diverges also.

\[
\text{ }
\]


\textbf{5. If $\sum{x_n}$ converges absolutely, show that $\sum{x_n^2}$ and $\sum{\frac{x_n}{1+x_n}}$ also converge absolutely.}

Consider the sequences $\left\{a_n\right\}=\left\{b_n\right\}=\left\{x_n\right\}$. Since both $\sum{a},\sum{b}$ converge absolutely, we know that their Cauchy inner product,
\[
c_n=\sum_{k=0}^{n}{a_kb_{n-k}}
\]
also converges absolutely (Theorem 23-9 in Spivak). We know that every subsequence of a convergent sequence also converges, so let us take the sequence $\left\{c_{2n}\right\}$. Then we can say that $x_n^2<c_{2n}$ for all $n$ since
\[
c_{2n}=\sum_{k=0}^{2n}{a_kb_{2n-k}}=\sum_{k=0}^{n-1}{a_kb_{2n-k}}+a_nb_n+\sum_{k=n+1}^{2n}{a_kb_{2n-k}}
\]

So we see that $c_{2n}>a_nb_n=x_n^2$. Thus $x_n^2$ is a smaller series than $c_{2n}$, which converges, and so it also must converge.

We can use a very similar argument for the second series. Consider $\left\{a_n\right\}=x_n$ and $\left\{b_n\right\}=\frac{1}{1+x_n}$. We know by hypothesis that $\sum{a_n}$ converges absolutely, and so as long as it doesn't converge to $0$, then $\sum{b_n}$ converges absolutely also. So we have two absolutely convergent series, meaning their Cauchy inner product,
\[
c_n=\sum_{k=0}^{n}{a_kb_{n-k}}=\sum_{k=0}^n{\frac{x_k}{1+x_{n-k}}}
\]
also converges absolutely. Yet again, this means that $\sum{c_{2n}}$ converges absolutely too. Again, we can say $\frac{x_n}{1+x_n}<c_{2n}$ for all $n$ since
\[
c_{2n}=\sum_{k=0}^{2n}{\frac{x_k}{1+x_{n-k}}}=\sum_{k=0}^{n-1}{\frac{x_k}{1+x_{n-k}}}+\frac{x_n}{1+x_n}+\sum_{k=n+1}^{2n}{\frac{x_k}{1+x_{n-k}}}
\] 
(and the rest of those terms are all positive). Thus since the desired series is smaller than an absolutely convergent series, it must also be absolutely convergent.

\textbf{6. For each series below, find all $x$ such that the series converges.}

\textbf{(a) $\displaystyle\sum_{n=1}^\infty{nx^n}$}

We examine the ratio
\[
\lim_{n\to\infty}{\left|\frac{(n+1)x^{n+1}}{nx^n}\right|=\lim_{n\to\infty}{\frac{(n+1)}{n}|x|}}=|x|\cdot\lim_{n\to\infty}{\frac{n+1}n}=|x|
\]

So this limit will be less than 1 (and thus have the series converge, when $|x|<1$. The only special case to consider is $|x|=1$, but in that case the series simplifies to $\sum n$ which clearly diverges. Thus the full solution is $|x|<1$.

\textbf{(b) $\displaystyle\sum_{n=1}^\infty{n^nx^n}$.}

We use the root test:
\[
\lim_{n\to\infty}{\sqrt[n]{(nx)^n}}=\lim_{n\to\infty}{nx}=x\lim_{n\to\infty}n
\]

It is easy to see that, regardless of what value we choose for $x$, the limit will diverge. Thus the only value of $x$ that will make it converge is $x=0$.

\textbf{(c) $\displaystyle\sum_{n=1}^\infty{\frac{n}{4^n}(2x-1)^n}$}

We use the ratio test again
\begin{align*}
\lim_{n\to\infty}{\left|\frac{\frac{n+1}{4^{n+1}}(2x-1)^{n+1}}{\frac{n}{4^n}(2x-1)^n}\right|}
=\lim_{n\to\infty}{\left|\frac{n+1}{4n}(2x-1)\right|}
=\frac14|2x-1|\lim_{n\to\infty}{\frac{n+1}{n}}
\end{align*}

As before, we know that $\frac{n+1}n\to1$, so we want
\[
\frac{|2x-1|}4<1\iff|2x-1|<4\iff\boxed{-\frac32<x<\frac52}
\]

\textbf{6. (d) $\displaystyle\sum_{n=1}^\infty{\frac{(n!)^p}{(pn)!}x^n}$.}

By the ratio test, we evaluate the limit
\[
\lim_{n\to\infty}{\left|\frac{\frac{((n+1)!)^p}{(p(n+1))!}x^{n+1}}{\frac{(n!)^p}{(pn)!}x^n}\right|}=
\lim_{n\to\infty}{\left|\frac{(pn!)(n+1)!^p}{(pn+p)!n!^p}x\right|}=
|x|\lim_{n\to\infty}{\frac{(n+1)^p}{(pn+p)\cdot(pn+p-1)\cdots(pn+1)}}
\]

We can write the denominator as
\[
|x|\lim_{n\to\infty}{\frac{(n+1)^p}{\displaystyle\prod_{k=1}^p{pn+k}}}=|x|\lim_{n\to\infty}{\frac{(n+1)^p}{\displaystyle\prod_{k=1}^p{p\left(n+\frac{k}{p}\right)}}}
\]

As $n\to\infty$, $\prod_{k=1}^p\left(n+\frac{k}{p}\right)=\left(n+\frac{k}{p}\right)^p\to(n+1)^p$, so it cancels with the numerator, and we are left with
\[
\frac{|x|}{\prod_{k=1}^p{p}}<1\implies\boxed{|x|<p^p}
\]
\end{document}