\documentclass[11pt]{amsart} \usepackage{geometry}
                 % ... or a4paper or a5paper or ...
% \geometry{landscape}                % Activate for for rotated page geometry
\usepackage[parfill]{parskip}
\usepackage{graphicx}
\usepackage{amssymb}
\usepackage{epstopdf}
\usepackage{fullpage}
\DeclareGraphicsRule{.tif}{png}{.png}{`convert #1 `dirname #1`/`basename #1 .tif`.png}

\title{PMATH 464 - Assignment 1}
\author{Adrian Petrescu (\#20240298)}
% \date{}                                           % Activate to display a given
% date or no date

\begin{document}

\maketitle
% \section{} \subsection{}

\textbf{1. Suppose that $X$ is an algebraic set and $L$ is a line in $A^n(k)$ such that $L\not\subset X$. Show that $L\cap X$ is either empty or a finite set of points.}

We assume that by a line in $\mathbb A^n(k)$, it is referred to a system of
$n-1$ equations of the form $Y=A_iX_i-B_i$, so that $L=V(X_2-(A_2X_1-B_2),
X_3-(A_3X_1-B_3), \ldots, X_n-(A_nX_1-B_n))$. Now, since
$L\not\subset X$, we cannot have $X$ be the whole space, and therefore there exists some non-zero
polynomial $F\in k[X_1,\ldots, X_n]$ such that $X\subseteq V(F)$. Thus if
$L \cap X$ were to be an infinite set, $L \cap V(F)$ would be an infinite set
as well. But
\begin{align*}
&V(X_2-(A_2X_1-B_2),
X_3-(A_3X_1-B_3), \ldots, X_n-(A_nX_1-B_n)) \cap V(F(X_1, \ldots,
X_n)) \\
=&F(X_1, A_2X_1-B_2, A_3X_1-B_3, \ldots, A_nX_1-B_n) \in k[X_1]\quad\text{since
$X_i-(A_iX_1+B_i)=0$}
\end{align*}

So $L\cap V(F)$ is the zero set of some polynomial in one variable. There are
obviously only finitely many points in such a set, so $L\cap X$ must be finite
or empty as well.

\textbf{2. Determine whether or not the following sets are algebraic.}

\textbf{(a) The set of points in $\mathbb R^2$ whose polar coordinates $(r,\theta)$ satisfy the equation $r=\theta$.}

The equation $r=\theta + 2k\pi$ for $k\in\mathbb N$ defines an Archimedean
spiral, so it will intersect \textit{any} line in $\mathbb R^2$ infinitely many
times. In polar coordinates, a line is specified by $\theta=a$, for some
$a\in\mathbb R$. So take $a=1$. Then $r=\theta+2k\pi$ intersects with $\theta=1$
at all the points $r=1+2k\pi$ for $k\in\mathbb N$. That is,
$r=1,1+2\pi,1+4\pi+\ldots$ are all points of intersection. Since there are
infinitely many of them, the spiral cannot possibly be an algebraic set.

\textbf{(b) The set of points in $\mathbb R^2$ whose polar coordinates $(r,\theta)$ satisfy $r=\cos{(\theta)}$.}



\textbf{(c) $\{(\cos{t},1,\sin{t}) \mid t\in\mathbb R\}\subseteq\mathbb R^3$.}

Intuitively speaking, this set looks like a single circle centered around the
point $(0,1,0)$ with radius $1$, parallel to the $xz$-plane; thus intuitively we
know it is the intersection of the vertical plane $y=1$ and the cylinder of
radius $1$ centered around $x$ axis, $x^2+z^2=1$, both of which are polynomials
in $\mathbb A^3(\mathbb R)$.

The $x$ and $z$ components parametrize the circle $x^2+z^2=1$ and the $y$
component specifies that the circle is on the plane $y=1$.

\textbf{(d) $\{(\cos{t},t,\sin{t}) \mid t\in\mathbb R\}\subseteq\mathbb R^3$.}

Consider the line in $\mathbb R^3$ parametrized by $\left\{\frac{1}{\sqrt2}, t,
\frac{1}{\sqrt2}\right\}$. It is easy to see that this line intersects the
given set at all points of the form $\left\{\frac{1}{\sqrt2},2k\pi,
\frac{1}{\sqrt2}\right\}$, for all $k\in Z$. Since there are infinitely many
such points of intersection, it cannot be an algebraic set.

\textbf{(e) $\{v\in\mathbb R^4 \mid |v|^2=1\}\subset\mathbb R^4$, where
$| \cdot |$ represents length.}

This set is simply $S^3$, the unit 3-sphere in 4-dimensional Euclidean space. It
is defined by $x^2+y^2+z^2+t^2=1$ which means that it is the zero set of the
polynomial $f(x,y,z,t)=x^2+y^2+z^2+t^2-1$.

\textbf{(f) $\{(z,w)\in\mathbb C^2 \mid |z|^2 + |w|^2 = 1\}\subset\mathbb C^2$,
where $|x+iy|^2=x^2+y^2$ for $x,y\in\mathbb R$.}

Observe that $\mathbb R^4\hookrightarrow\mathbb C^2$; that is, $\mathbb R^4$
can be embedded into $\mathbb C^2$ with the ring homomorphism
$\phi(a,b,c,d)=(a+bi,c+di)$. In particular, this means that $\phi$ carries zero
sets in $\mathbb R^4$ to zero sets in $\mathbb C^2$. Now, we proved in part (e)
that $\{v\in\mathbb R^4 \mid |v|^2=1\}\subset\mathbb R^4$ was the zero set of
$x^2+y^2+z^2+t^2-1$.  But $\phi(v)=(z,w)=(x+yi, z+ti)$, so $|v|=1\implies
x^2+y^2+z^2+t^2=1\implies |z|^2+|w|^2=1$. Since $\phi$ carries zero sets to
zero sets, this must also be an algebraic set.

\textbf{3. Consider the real line $\mathbb R$ endowed with the Zariski topology. Verify that, given any two points $p,q\in\mathbb R$, any two open neighbourhoods of $p$ and $q$, have a non-empty intersection, thus proving that the Zariski topology on $\mathbb R$ is not Hausdorff.}

On $\mathbb R$, the only algebraic sets (and therefore the only closed sets) are
empty, finite, or $\mathbb R$ itself. Let $X\subseteq\mathbb R$ be a closed set
in $\mathbb R$. Therefore the only open sets are $\mathbb R$ with finitely many
points missing. Since the neighbourhoods of $p$ and $q$ contain open sets, their
intersection will also be an open set -- namely $\mathbb R$ missing the points
which both $N(p)$ and $N(q)$ were missing. This is nonempty, and therefore the
Zariski topology on $\mathbb R$ is not Hausdorff.

\textbf{If $k$ is a finite field, show that every subset of $\mathbb A^n{k}$ is both open and closed in the Zariski topology. Is the Zariski topology Hausdorff in this case?}

We know that if $k$ is a finite field, any subset of $A^n(k)$ is an algebraic
set. Indeed, if $\{v_0,v_1,\ldots,v_m\}$ is a subset of $A^n(k)$ then it is a
finite union $\cup_{i=0}^m{(v_i)}$. Thus it suffices to prove that each $v_i$ is
an algebraic set. Well, if $v_i=(a_1,a_2,\ldots,a_n)$ then we have
$V(X_1-a_1,X_2-a_2,\ldots,X_n-a_n)=v_i$, so each $v_i$ \textit{is} an algebraic
set.

Once we've established that all subsets are algebraic, it follows that all
subsets are both closed (since they're algebraic) and open (since their
complement is also a subset of $A^n(k)$ which must also be algebraic, hence
closed). Therefore the Zariski topology over such fields is Hausdorff, since for
any distinct points $p,q$, the sets $\{p\}$ and $\{q\}$ are open subsets with
empty intersection.

\textbf{5. Let $k$ be any field and $M_{n\times n}(k)$ be the set of $n\times n$ matrices with entries in $k$. This set can naturally be identified with $\mathbb A^{n^2}(k)$, by considering the (ordered) $n^2$ entries of a matrix as a point in $\mathbb A^{n^2}(k)$, and is thus endowed with the Zariski topology. Show that the group $GL(n,k)$ of all invertible matrices in $M_{n\times n}(k)$ is open in the Zariski topology on $M_{n\times n}(k)$.}

To show the open-ness of $GL(n,k)$, we must show that its complement (the
non-invertible matrices) are an algebraic set. We know that the non-invertible
matrices are precisely those with determinant $0$. However, the determinant of a
matrix is itself a polynomial $\det_n:\mathbb A^{n^2}(k)\to k$, and it will be
$0$ precisely when it is passed a (non-invertable) matrix of determinant 0. Thus
the complement of $GL(n,k)$ is precisely $V(\det_n)$, a closed set.

\textbf{6. Let $V\subset\mathbb A^n(k)$ and $W\subset\mathbb A^m(k)$ be algebraic
sets. Show that \[ V\times W:=\{(a_1,\ldots,a_n,b_1,\ldots,b_m) \mid
(a_1,\ldots,a_n)\in V, (b_1,\ldots,b_m)\in W\} \] is an algebraic set in $\mathbb
A^{n+m}(k)$. It is called the \textit{product} of $V$ and $W$.}

Since $V,W$ are algebraic, there exist two finite sets of polynomials $S_1, S_2$
such that $V(S_1)=V$ and $V(S_2)=W$. Now, we note that for each polynomial
$f(a_1,\ldots,a_n)\in\mathbb A^n(k)$ there is a corresponding polynomial
$f'\in\mathbb A^{n+m}(k)$ such that
$f'(a_1,a_2,\ldots,a_n,x_1,\ldots,x_{m})=f(a_1,a_2,\ldots,a_n), \forall
x_1,\ldots,x_m\in k$, and a corresponding polynomial
$f''(x_1,\ldots,x_m,a_1,\ldots,a_n)=f(a_1,\ldots,a_n),\forall x_1,\ldots,x_m\in
k$. Thus we can safely talk about the sets $S_1',S_2''$ as being subsets of $k[X_1,\ldots,X_{n+m}]$.


Let $I=I(S_1')$ and $J=I(S_2'')$. Then consider the ideal
$I\cup J$. We claim that $V(I\cup J)=V\times W$.

It is easy to see that $V\times W \subseteq V(I\cup J)$ since for any $r=(a_1,\ldots,a_n)\in V, (b_1,\ldots,l_m)\in W$ and any $f\in I\cup J$, we have 
$f(a_1,\ldots,a_n,b_1,\ldots,b_m)$ as the sum of multiples of things in $S_1', S_2''$. Since polynomials in $S_1'$ only care about the first $n$ variables and polynomials in $S_2''$ only care about the last $m$ variables, each such polynomial will be $0$ at $r$, and thus so will its linear combinations. 

To show $V(I\cup J)\subseteq V\times W$,  let $r=(x_1,\ldots,x_n,y_1,\ldots,y_m)\in V(I\cup J)$. Then for any $f'\in I$ we will have a corresponding $f\in S_1$ such that $f'(r)=f(x_1,\ldots,x_n)=0$, so $(x_1,\ldots,x_n)\in V$. Similarly $(y_1,\ldots, y_m)\in W$. Therefore $r=(x_1,\ldots,x_n,y_1,\ldots,y_m)\in V\times W$.

Thus $V\times W=V(I\cup J)$ is an algebraic set.

\textbf{7. The \textit{product topology} on the Cartesian product $X\times Y$ of two topologoical spaces is defined as the topology whose open sets are the unions of subsets $A\times B$, where $A,B$ are open subsets of $X$ and $Y$ respectively. Let $k$ be any field, and consider the affine line $\mathbb A^1(k)$ and the affine plane $\mathbb A^2(k)$.}

\textbf{(a) Describe the product topology of the Zariski topologies on the two copies of $\mathbb A^1$ in $\mathbb A^1\times \mathbb A^1$.}

We recall that the closed sets in $\mathbb A^1(k)$ are the empty set, the entire line, and finite sets of points, so the open sets are the empty set, the entire line, and the line with finitely many points missing. Now, suppose $X,Y\in\mathbb A^1$ were open sets missing $(a_1,\ldots,a_n)$ and $(b_1,\ldots,b_m)$ respectively. Then $X\times Y$ would be missing the whole plane except for horizontal lines of the form $(x,k)$ for $x\in a_i, 1\leq i\leq n$, and vertical lines of the for $(k, y)$ for $y=b_j, 1\leq j\leq m$.

Thus in $\mathbb A^1\times\mathbb A^1$ the closed sets are the empty set, the entire plane, and finite collections of horizontal and vertical lines.

\textbf{(b) Show that although $\mathbb A^2$ can be naturally identified with $\mathbb A^1\times\mathbb A^1$, the Zariski topology on $\mathbb A^2$ is not equal to the "product Zariski topology" on $\mathbb A^1\times\mathbb A^1$ you described in (a), if $k$ is an infinite field.}

Any closed set in the Zariski topology on $\mathbb A^2$ not of the form of a vertical or horizontal line is not a closed set in the product Zariski topology. For example, by the classification given in part (a) above, the set $V(Y-X)\subset\mathbb A^2$ is closed under the Zariski topology, but not under the product Zariski topology, so they cannot be equal.
\end{document}
