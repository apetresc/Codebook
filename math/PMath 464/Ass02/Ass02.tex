\documentclass[11pt]{amsart}
\usepackage{geometry}                % See geometry.pdf to learn the layout options. There are lots.
\geometry{letterpaper}                   % ... or a4paper or a5paper or ... 
%\geometry{landscape}                % Activate for for rotated page geometry
\usepackage[parfill]{parskip}    % Activate to begin paragraphs with an empty line rather than an indent
\usepackage{graphicx}
\usepackage{fullpage}
\usepackage{amssymb}
\usepackage{epstopdf}
\usepackage{amsthm}
\DeclareGraphicsRule{.tif}{png}{.png}{`convert #1 `dirname #1`/`basename #1 .tif`.png}

\theoremstyle{definition}
\newtheorem{thm}{Theorem}
\newtheorem*{defn}{Definition}

\title{PMATH 464 - Assignment 2}
\author{Adrian Petrescu (\#20240298)}
%\date{}                                           % Activate to display a given date or no date

\begin{document}
\maketitle
%\section{}
%\subsection{}

\textbf{1. Let $k$ be an infinite field. Prove that any non-empty Zariski open subset of $\mathbb A^n(k)$ is dense.}

There are many equivalent definitions of density; we will use the following:

\begin{defn}
A subset $A$ of a topological space $X$ is called {\bf dense} if for any point $x$ in $X$, any neighborhood of $x$ contains at least one point from $A$. Equivalently, $A$ is dense if, for any open subset $Y\subseteq X$, $A\cap Y\not=\emptyset$.
\end{defn}

%So let $A$ be any non-empty Zariski open subset of $\mathbb A^n(k)$. Suppose there was some other open subset $B\subset\mathbb A^n(k)$ such that $A\cap B=\emptyset$. 	Then let $\bar{A},\bar{B}$ be the (closed) complements of $A,B$ respectively. Since $\bar{A},\bar{B}$ are closed, there exist finite sets $S_1,S_2\subseteq k[x_1,x_2,\ldots,x_n]$ such that $V(S_1)=\bar{A}$ and $V(S_2)=\bar{B}$.

%Since $A\cap B=\emptyset$, it follows that $B\subset\bar{A}=V(S_1)$; so every polynomial in $S_1$ is zero on $B$.

So, let $A$ be any non-empty Zariski open subset of $\mathbb A^n(k)$. We define $\bar A$ to be the (closed) complement of $A$. Suppose that there was some other open subset $B\subset\mathbb A^n(k)$ such that $A\cap B=\emptyset$; this would imply that $B\subset \bar A$.

Then, let $L$ be any line in $\mathbb A^n(k)$. Since $\bar A$ is closed, we know that $L\cap\bar A$ is a finite collection of points, so since $B\subset\bar A$, this implies $L\cap B$ is finite as well. But we know that $B\cup\bar B=\mathbb A^n(k)$, so if $L\cap B$ is finite, it must be the case that $L\cap\bar B$ is infinite (since the line $L$ has an infinite number of points, and $k$ is infinite). But $B$ is open implies that $\bar B$ is closed, which is a contradiction.

\textbf{2. In this exercise, we prove that the spiral $r=\theta$ is dense in $\mathbb R^2$.}

\textbf{(a) Let $V$ and $W$ be two algebraic sets in $\mathbb A^n(k)$. Show that $V=W$ if and only if $I(V)=I(W)$.}

The forward direction of this is obvious -- if $V=W$ then $I(V)=I(W)$ since the operation $I$ is well-defined. Indeed, suppose $V=W$ but $I(V)\not=I(W)$. Without loss of generality, assume $I(V)\not\subseteq I(W)$. Let $f\in I(V)-I(W)$. Since it is in $I(V)$, it is zero over every point in $V$, which also means it is zero over every point in $W$, which implies that it is in $I(W)$, a contradiction.

Next, assume $I(V)=I(W)$, and suppose $v\in V$. Then $f(v)=0, \forall f\in I(V)$; but by the same token this means $f(v)=0,\forall f\in I(W)$ which by definition means $v\in W$. Thus $V\subseteq W$. An identical argument on $w\in W$ shows that $W\subseteq V$ from which we conclude that $V=W$.

\textbf{(b) Let $k$ be an infinite field and $f\in k[x,y]$ be an irreducible polynomial such that $V(f)$ is an infinite set of points in $\mathbb A^2(k)$. Prove that if $E$ is any infinite subset of $V(f)$, then $\bar E=V(f)$ in the Zariski topology.}

Since both $\bar E$ and $V(f)$ are algebraic sets by definition, it suffices to show that $I(\bar E)=I(V(f))=\langle f\rangle$. Well, since $\bar E$ is the intersection of all closed sets containing $E$, this means in particular that $\bar E\subseteq V(f)$; thus $f(p)=0, 	\forall p\in\bar E$ so $f\in I(\bar E)$ implying that $\langle f\rangle\subseteq I(\bar E)$. To show that $I(\bar E)\subseteq\langle f\rangle$, just note that if $I(\bar E)=\langle f,g\rangle$ then $\bar E=V(f)\cap V(g)$; since $\bar E$ is infinite this implies that $V(f)$ and $V(g)$ have an infinite intersection; but $f,g$ are irreducible so their intersection is at most a finite set of points. Thus it must be the case that $I(\bar E)\subseteq \langle f\rangle$, and we are done.

\textbf{(c) Let $E$ be the curve in $\mathbb R^2$ given in polar coordinates by $r=\theta$. Show that $E$ is dense in $\mathbb R^2$ in the Zariski topology.}

This time we will use the following (equivalent) definition of density:
\begin{defn}
A subset $A$ of a topological space $X$ is called {\bf dense} if the closure $\bar A$ is equal to $X$.
\end{defn}

\textbf{4. Let $V$ and $W$ be algebraic sets in $\mathbb A^n(k)$ such that $V\subset W$. Show that each irreducible component of $V$ is contained in some irreducible component of $W$.}

We can write
\[ V = V_1 \cup V_2 \cup \cdots \cup V_j \]
and
\[ W = W_1 \cup W_2 \cup \cdots \cup W_k \]

We proceed in a way analogous to the proof of the uniqueness of an irreducible decomposition. For any $V_i$ we know that $V_i=\bigcup_j{(W_j\cap V_i)}$. Now, since all the $V_i$'s and $W_j$'s are irreducible and closed, it must be the case that $V_i$ is entirely contained in just one of the $(W_j\cap V_i)$ components (otherwise, since the intersection of closed sets is closed, we would have contradicted the irreducibility of $V_i$). Thus $V_i\subseteq(W_j\cap V_i)\implies V_i\subseteq W_j$. This is true for each $V_i$ so the statement holds.

\textbf{5. Show that although the algebraic curve $X=V(y^2+x^2(x-1)^2(x^2+1))$ is irreducible in $\mathbb C^2$, it is reducible in $\mathbb R^2$. What does this imply about the ideal of $X$? Verify your answer by giving an explicit description of $I(X)$ in both $\mathbb R[x,y]$ and $\mathbb C[x,y]$.}

We will consider the real case first. We see that $y^2$ and $x^2(x-1)^2(x^2+1)$ are both non-negative quantities. Therefore, if their sum is $0$, they must both be $0$. That is, $y^2=0\implies y=0$ and $x^2(x-1)^2(x^2+1)=0\implies x=0$ or $x=1$. Thus there are two irreducible components: $\{0,0\}$ and $\{0,1\}$.

In $\mathbb C^2$, however, this is an infinite set. We claim it is irreducible. By the classification of plane curves in $\mathbb A^2(k)$, it suffices to show that the polynomial $y^2+x^2(x-1)^2(x^2+1)$ is irreducible in $\mathbb C[x,y]$. Well, suppose it was reducible. Then we have two cases: either it factors as $(y^2+f(x))g(x)$ or as $(y+f(x))(y+g(x))$. In the first case, we have
\[
(y^2+f(x))g(x) = g(x)y^2 + f(x)g(x)
\]
But since the coefficient on $y^2$ needs to be $1$ we get that $g(x)=1$ which is a constant polynomial; thus this is not a proper factorization.

In the second case we have
\[
(y+f(x))(y+g(x)) = y^2 + (f(x) + g(x))y + f(x)g(x)
\]
Since there is no $y$ term in our polynomial, we must have $f(x)+g(x)=0\implies f(x)=-g(x)$. This means that our constant (with respect to $y$) term must be of the form $-f(x)^2$. Thus, $f(x)^2=-x^2(x-1)^2(x^2+1)$. Now, multiplying a non-square by a square will never create a square; thus this implies that since $-1$, $x^2$ and $(x-1)^2$ are perfect squares, $x^2+1$ must be a perfect square as well. But the factorization $x^2+1=(x-i)(x+i)$ is clearly not a square, so this factorization cannot exist. Thus the polynomial is irreducible, which by our classification, means that $X$ is irreducible in $\mathbb C^2$.

Now consider the ideal $I(X)$ in $\mathbb R[x,y]$. It is a finite collection of points so its ideal looks like 

\textbf{6. Let $k$ be an algebraically closed field, which may have characteristic $2$. Find the irreducible components of the following algebraic sets.}

\textbf{(a) $V(x^3+x^2y-2xy-2y^2)$ in $\mathbb A^2(k)$.}

We notice that the polynomial factors as $(x+y)(x^2-2y)$. This expression will be $0$ whenever either of those two factors is $0$ so this implies that the zero set will look like the line $y=-x$ and the parabola $y=\frac{x^2}{2}$.


If the field is of characteristic $2$, then immediately the terms $-2xy$ and $-2y^2$ disappear, leaving $V(x^3+x^2y)=V(x^2(x+y))$. Well, $x^2=0$ only when $x=0$; so the vertical plane $x=0$ is one irreducible component. That leaves $x+y=0\implies y=-x$ which includes the diagonal plane $y=-x$. So the irreducible components become $V(x)\cup V(x+y)$. 

\textbf{(d) $V(x^2-yz,xz-x)$ in $\mathbb A^3(k)$.}

We see that $V(x^2-yz,xz-z)=V(x^2-yz)\cap V(xz-z)$. Thus for any point in the intersection, we must have $xz-z=z(x-1)=0\implies z=0$ or $x-1$. If $z=0$ then $x^2-yz=x^2=0\iff x=0$, so we have the plane $x=0$. Otherwise, if $x=1$, then $x^2-yz=0\iff yz=1\iff y=\frac{1}{z}$. This induces the zero set $yz-1$. Thus the irreducible components are $V(x^2-yz,xz-x)=V(x)\cup V(yz-1)$. 

In the case of a field of characteristic $2$,

\textbf{7. Let $k$ be any field. Are the following ideals prime, radical, or closed in $k[x,y]$?}

\textbf{(b) $\langle x+y, xy \rangle$}

The only point in $V(\langle x+y,xy\rangle)$ is $(0,0)$ since $x+y$ is the line $y=-x$ passing through the origin, and $xy=0$ are the $x$- and $y$- axes; these three curves all meet only at the origin. Therefore the ideal of $(0,0)$ consists of all polynomials without a constant $a_0$ term. It is clear that $\langle x+y,xy\rangle$ does not generate all these: for instance, $y$ is in $I(\{0,0\})$ but not in $\langle x+y, xy\rangle$. Thus it is not closed. Moreover, $I(V(I))=I(\{0,0\})=\operatorname{Rad}{(I)}\not=I$ by Hilbert's Nullstellensatz, so it is also not a radical ideal. Lastly, it is not prime since $xy\in\langle x+y,xy\rangle$ but neither $x$ nor $y$ is.

\textbf{(c) $\langle x^2-y^2\rangle$}

Call this ideal $I$. We see that $V(I)$ is the simply the set $V(x+y,x-y)$, since $x^2-y^2=0\implies y=\pm\sqrt{x^2}$. So do we have $I(V(I))=I$? Well, consider the ideal $\langle x+y,x-y\rangle$. It's clear that $\langle x^2-y^2\rangle\subseteq\langle x+y,x-y\rangle$ since $x^2-y^2=(x+y)(x-y)\in\langle x+y,x-y\rangle$. However, the converse does not hold; $x+y$ is not in $\langle x^2-y^2\rangle$, so $I$ is not the smallest ideal whose zero set contains $V(I)$. Thus it is not algebraically closed. Similarly, we have that $\operatorname{Rad}{(I)}\not=I$ by Hilbert's Nullstellensatz as before. Finally, it is not prime either, as $(x+y)(x-y)\in I$ but neither $x+y$ nor $x-y$ are.

\end{document}  