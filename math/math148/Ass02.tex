\documentclass[11pt]{article}
\usepackage{geometry}                % See geometry.pdf to learn the layout options. There are lots.
\geometry{letterpaper}                   % ... or a4paper or a5paper or ... 
%\geometry{landscape}                % Activate for for rotated page geometry
\usepackage{fullpage}
\usepackage[parfill]{parskip}    % Activate to begin paragraphs with an empty line rather than an indent
\usepackage{graphicx}
\usepackage{amssymb}
\usepackage{epstopdf}
\usepackage{amsmath}
\DeclareGraphicsRule{.tif}{png}{.png}{`convert #1 `dirname #1`/`basename #1 .tif`.png}

\title{Math 148 � Assignment 2}
\author{Adrian Petrescu (\#20240298)}
%\date{}                                           % Activate to display a given date or no date

\begin{document}
\maketitle
%\section{}
%\subsection{}
\textbf{1. Suppose that $f$ is continuous and increasing over $[0,1]$ and that $f(0)=0$ while $f(1)=1$. If $g$ is the inverse function of $f$, why is $g$ integrable over $[0,1]$? Without doing any calculations whatsoever, find $\displaystyle\int_0^1f+\int_0^1g$.}

Since $f$ is increasing, it is one-to-one; $f$ is also continuous, and so these two conditions guarantee that $g$ is also continuous and therefore integrable. We know $g(0)=0$ and $g(1)=1$ also.


\textbf{2. (a) We saw that on any interval, functions with a bounded derivative are uniformly continuous. Is there a function on some interval with an unbounded derivative that is still uniformly continuous?}

\textbf{(b) Is there a non-differentiable function that is still uniformly continuous?}

\textbf{3. Find the area in the first quadrant enclosed by the curves $y=\frac1{1+x^2}$ and $y=x$. This is a routine exercise using the fundamental theorem.}

We begin by finding the point of intersection between $f(x)=\frac1{1+x^2}$ and $g(x)=x$. Of course, solving cubics on our own is not very fun, but that's what we pay Maple for. We find that these two functions intersect at $(0.6823,0.6823)$.  On the interval $[0,0.6823]$, it is easy to see that $f$ dominates $g$ (since $f(0)=1$ while $g(0)=0$). They only intersect once, however, so on the interval $[0.06823,\infty)$ we know that $g$ dominates $f$. So we can reduce this problem into \[ \int_0^{0.6823}g+\int_{0.6823}^\infty f\] Obviously $\int_0^{0.6823}g=\frac{x^2}{2}\displaystyle\vert_0^{0.6823}=0.2328$ We also know by inspection that $\int\frac1{1+x^2}$ is simply $\arctan{x}$,  and so $\int_{0.6823}^\infty f=\arctan{x}\vert_{0.6823}^\infty$. It is well known that $\lim_{x\to\infty}{\arctan{x}}=\frac\pi2$, and we evaluate $\arctan{0.6823}=0.5987$ and so $\int_{0.6823}^\infty f=0.972$. Finally we have
\begin{align*}
&\int_0^{0.6823}g+\int_{0.6823}^\infty f\\
=&0.2328+\frac\pi2-0.5987=\boxed{1.205}
\end{align*}

\textbf{4. The functions $y=\sin{x}$ and $y=\cos{x}$ intersect each other infinitely often. Find the area of the region enclosed by these functions between any two adjacent intersections.}

By recalling the unit circle definition of the $\sin$ and $\cos$ functions, we see that the unit circle intersects the line $y=x$ in two places: $\sin\left(\frac\pi4\right)=\cos{\left(\frac\pi4\right)}=\frac1{\sqrt2}$, and then again at $\sin{\left(\frac{3\pi}4\right)}=\cos{\left(\frac{3\pi}4\right)}=-\frac1{\sqrt2}$ So:
\begin{align*}
&\int_{\frac\pi4}^{\frac{3\pi}4}{\sin{x}-\cos{x}}\\
=&\int_{\frac\pi4}^{\frac{3\pi}4}{\sin{x}}-\int_{\frac\pi4}^{\frac{3\pi}4}{\cos{x}}\\
=&-\cos{x}\big|_{\frac\pi4}^{\frac{3\pi}4}-\sin{x}\big|_{\frac\pi4}^{\frac{3\pi}4}\\
=&\cos{\left(\frac\pi4\right)}+\sin{\left(\frac\pi4\right)}-\cos{\left(\frac{3\pi}4\right)}-\sin{\left(\frac{3\pi}4\right)}\\
=&\frac4{\sqrt2}
\end{align*}

\textbf{5. We know, by the method of lifting, that \[x-\frac{x^3}{6}\leq\sin x\leq x-\frac{x^3}{6}+\frac{x^5}{120},\] for all $x\geq0$. Exploit this information to prove that $\displaystyle\int_0^\frac12\sin{(x^2)}dx\approx\frac{111}{2688}$ with an error that is no more than $\frac1{2703360}$.} 

From the inequality given in the problem, we can infer that
\[x^2-\frac{x^2}{6}\leq\sin{x^2}\leq x^2-\frac{x^2}{6}+\frac{x^{10}}{120}\]
Taking the integral over the interval $[0,\frac12]$ of all sides, we get
\[\int_0^{\frac12}{x^2}-\int_0^{\frac12}{\frac{x^6}{6}}\leq\int_0^{\frac12}{\sin{x^2}}\leq\int_0^{\frac12}{x^2}-\int_0^{\frac12}{\frac{x^6}6}+\int_0^{\frac12}{\frac{x^{10}}{120}}\]
These are all very easy integrals to evaluate and we soon see
\[\frac1{24}-\frac{1}{5376}\leq\int_0^{\frac12}{\sin{x^2}}\leq\frac1{24}-\frac{1}{5376}+\frac{1}{2703360}\]
\[\boxed{\frac{223}{5376}\leq\int_0^{\frac12}{\sin{x^2}}\leq\frac{784967}{18923520}}\]
\textbf{6. If $f$ is integrable over $[a,b]$, show that $|f|$ is also integrable over $[a,b]$. Then show that $\left|\int_a^bf\right|\leq\int_a^b|f|$. This important fact is known as the triangle theorem for integration.}

\textbf{7. Let $a<c<d<b$ and suppose $f$ is the function that takes the constant value $1$ on $[c,d]$ and the value $0$ on $[a,c)$ and the value $0$ on $(d,b]$. Prove that $f$ is integrable, and calculate $\int_a^bf$.}

\textbf{8. Suppose that $f$ is continuous over $[a,b]$, that $f(x)\geq0$ for all $x\in[a,b]$, and that $\displaystyle\int_a^bf=0$. Prove that $f(x)=0$ for all $x\in[a,b]$.}
\end{document}  